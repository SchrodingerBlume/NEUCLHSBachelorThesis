% ==================================== %
% =======  排版(typeset)设置  ======== %
% ==  已经调得很接近Word了,建议不要动  == %
% ==================================== %

% =========================== %
% ======  中西文字距设置  ====== %
% =========================== %
\xeCJKsetup{CJKecglue=\hskip 0.25em minus 0.25em}

% =========================== %
% ======  中文字距设置  ====== %
% =========================== %
% \ziju{} % 在V-1.0.2中恢复了默认字距,此命令接受一个浮点数值

% ================================= %
% =========  全角符号宽度  ========== %
% ======  这部分我没做太多测试  ====== %
% ==  不满意可以调或者全删了恢复默认 == %
% ================================= %

% 符号宽度
\xeCJKsetwidth{。;}{1.06em} 
\xeCJKsetwidth{,}{0.8em}
\xeCJKsetwidth{、}{0.78em}
\xeCJKsetwidth{()‘’“”}{1em}

% 左边符号碰上右边符号时二者的整体宽度
\xeCJKsetkern{)}{,}{0.5em}
\xeCJKsetkern{)}{。}{0.5em}
\xeCJKsetkern{)}{!}{0.5em}
\xeCJKsetkern{)}{?}{0.5em}
\xeCJKsetkern{’}{,}{0.5em}
\xeCJKsetkern{’}{。}{0.5em}
\xeCJKsetkern{’}{!}{0.5em}
\xeCJKsetkern{’}{?}{0.5em}
\xeCJKsetkern{”}{,}{0.5em}
\xeCJKsetkern{”}{。}{0.5em}
\xeCJKsetkern{”}{!}{0.5em}
\xeCJKsetkern{”}{?}{0.5em}
\xeCJKsetkern{:}{“}{0.5em}
\xeCJKsetkern{:}{‘}{0.5em}
\xeCJKsetkern{;}{“}{0.5em}
\xeCJKsetkern{;}{’}{0.5em}
\xeCJKsetkern{!}{”}{0.5em}
\xeCJKsetkern{!}{’}{0.5em}
\xeCJKsetkern{?}{”}{0.5em}
\xeCJKsetkern{?}{’}{0.5em}
\xeCJKsetkern{(}{“}{0.5em}
\xeCJKsetkern{”}{)}{0.5em}

% =========================== %
% ======  西文字距设置  ====== %
% =====  也是最好不要动  ===== %
% =========================== %
\fontdimen2\font=0.275em     % 字间空格
\fontdimen3\font=0.1375em    % 拉伸
\fontdimen4\font=0.0917em    % 压缩
\fontdimen7\font=0em         % 无句末额外空格
\spaceskip=0.25em plus 0.03em minus 0.07em

% =========================== %
% ======  西文断字设置  ====== %
% =====  默认完全不断字  ===== %
% ====  这个可以按需要调整  ==== %
% =========================== %
\tolerance=1
\emergencystretch=\maxdimen
\hyphenpenalty=10000 % hyphenpenalty越大断字出现的就越少
\hbadness=10000 % tolerance越大,换行就会越少,即把本该断开放到下一行的单词,完整地留在当前行

% 西文断字词典
% 如果遇到了断字不对的情况,可以手动指定某单词如何断字
% 例如以下命令使hyphenation只能从标有短横线的地方断开
\hyphenation{hy-phen-a-tion}