\chapter{附录1:中文译文}

\ftitle{在这输入英文文献翻译后的中文标题}

\fauthor{%
Yushan Qiu\thanksA{1}, Yahong Zhang\thanksA{2}, Liwen Tian\thanksA{2}, Quan Zou\thanksA{3}, Pu Zhao\thanksA{2}\corrauth
}

\thanksB{1}深圳大学数学与统计学院,深圳 518060

\thanksB{2}东北大学生命科学与健康学院,沈阳 110819

\thanksB{3}电子科技大学基础与前沿研究院,成都 610056

\corrinfo{zhaopu6687700@163.com}

\begin{cfabstract} % 附录中文摘要
在这里输入您翻译后的摘要:

上皮间质转化(epithelial-to-mesenchymal transition,EMT)是肿瘤转移的潜在机制,显示出肿瘤细胞的转移潜力。尽管EMT的转录调控已得到充分研究,但选择性剪接 (alternative splicing,AS) 调控在EMT中的作用在很大程度上仍未得到表征。RNA-seq数据集的快速积累为开发将mRNA亚型变异与EMT相关联的计算方法提供了机会。在这项研究中,我们提出了正则化模型来识别EMT期间的显着AS事件。我们的实验结果证实,预测的AS事件与EMT期间必不可少的细胞死亡、粘连斑-内联足移位和紧密连接形成密切相关。因此,我们的研究强调了EMT期间转录后调节的广泛作用,并确定了作为不同调节节点的AS事件的关键子集。

\end{cfabstract}

% 附录中文关键词
\cfkeywords{分子生物学;细胞生物学;癌症}

\begin{efabstract} % 附录英文摘要
在这里输入摘要原文:

Epithelial-to-mesenchymal transition (EMT) is the underlying mechanism for tumor metastasis and shows the metastatic potential of tumor cells. Although the transcriptional regulation of EMT has been well studied, the role of alternative splicing (AS) regulation in EMT remains largely uncharacterized. The rapid accumulation of RNA-seq datasets has provided the opportunities for developing computational methods to associate mRNA isoform variations with EMT. In this study, we propose regularization models to identify significant AS events during EMT. Our experimental results confirm that the predicted AS events are closely related to apoptosis, focal adhesion-invadopodium shift and tight junction formation that are essential during EMT. Therefore, our study highlights the broad role of posttranscriptional regulation during EMT and identifies key subsets of AS events serving as distinct regulatory nodes.
\end{efabstract}

% 附录英文关键词
\efkeywords{Molecular biology; Cell biology; Cancer}

% ==============================
% ==    以下开始撰写附录1正文    ==
% ==============================

\fchapter{希腊字母快捷命令一览\footnote{如未特殊声明,本章所有命令均只能在数学模式中使用。}}\label{greekcommand}

\fsection{小写希腊字母——包括24种常规体字母和7种变体字母}



\fbiaoge
{小写希腊字母命令}{tab:greek}{9}{Y}
{名称 & \makecell[c]{默认\\斜体} & \fangsong{命令} & \textbf{\textit{粗斜体}} & \fangsong{命令} & 正体 & \fangsong{命令} & \textbf{粗正体} & \fangsong{命令}}
{alpha & $\alpha$ & \verb|\alpha| & $\bfalpha$ & \verb|\bfalpha| & $\upalpha$ & \verb|\upalpha| & $\bfupalpha$ & \verb|\bfupalpha| \\
beta & $\beta$ & \verb|\beta| & $\bfbeta$ & \verb|\bfbeta| & $\upbeta$ & \verb|\upbeta| & $\bfupbeta$ & \verb|\bfupbeta| \\
gamma & $\gamma$ & \verb|\gamma| & $\bfgamma$ & \verb|\bfgamma| & $\upgamma$ & \verb|\upgamma| & $\bfupgamma$ & \verb|\bfupgamma| \\
delta & $\delta$ & \verb|\delta| & $\bfdelta$ & \verb|\bfdelta| & $\updelta$ & \verb|\updelta| & $\bfupdelta$ & \verb|\bfupdelta| \\
epsilon & $\epsilon$ & \verb|\epsilon| & $\bfepsilon$ & \verb|\bfepsilon| & $\upepsilon$ & \verb|\upepsilon| & $\bfupepsilon$ & \verb|\bfupepsilon| \\
\kaishu{varepsilon} & $\varepsilon$ & \verb|\varepsilon| & $\bfvarepsilon$ & {\zihao{6}\verb|\bfvarepsilon|} & $\upvarepsilon$ & {\zihao{6}\verb|\upvarepsilon|} & $\bfupvarepsilon$ & \verb|\bfupvarepsilon| \\
zeta & $\zeta$ & \verb|\zeta| & $\bfzeta$ & \verb|\bfzeta| & $\upzeta$ & \verb|\upzeta| & $\bfupzeta$ & \verb|\bfupzeta| \\
eta & $\eta$ & \verb|\eta| & $\bfeta$ & \verb|\bfeta| & $\upeta$ & \verb|\upeta| & $\bfupeta$ & \verb|\bfupeta| \\
theta & $\theta$ & \verb|\theta| & $\bftheta$ & \verb|\bftheta| & $\uptheta$ & \verb|\uptheta| & $\bfuptheta$ & \verb|\bfuptheta| \\
\kaishu{vartheta} & $\vartheta$ & \verb|\vartheta| & $\bfvartheta$ & \verb|\bfvartheta| & $\upvartheta$ & \verb|\upvartheta| & $\bfupvartheta$ & \verb|\bfupvartheta| \\
iota & $\iota$ & \verb|\iota| & $\bfiota$ & \verb|\bfiota| & $\upiota$ & \verb|\upiota| & $\bfupiota$ & \verb|\bfupiota| \\
kappa & $\kappa$ & \verb|\kappa| & $\bfkappa$ & \verb|\bfkappa| & $\upkappa$ & \verb|\upkappa| & $\bfupkappa$ & \verb|\bfupkappa| \\
\kaishu{varkappa} & $\varkappa$ & \verb|\varkappa| & $\bfvarkappa$ & \verb|\bfvarkappa| & $\upvarkappa$ & \verb|\upvarkappa| & $\bfupvarkappa$ & \verb|\bfupvarkappa| \\
lambda & $\lambda$ & \verb|\lambda| & $\bflambda$ & \verb|\bflambda| & $\uplambda$ & \verb|\uplambda| & $\bfuplambda$ & \verb|\bfuplambda| \\
mu & $\mu$ & \verb|\mu| & $\bfmu$ & \verb|\bfmu| & $\upmu$ & \verb|\upmu| & $\bfupmu$ & \verb|\bfupmu| \\
nu & $\nu$ & \verb|\nu| & $\bfnu$ & \verb|\bfnu| & $\upnu$ & \verb|\upnu| & $\bfupnu$ & \verb|\bfupnu| \\
xi & $\xi$ & \verb|\xi| & $\bfxi$ & \verb|\bfxi| & $\upxi$ & \verb|\upxi| & $\bfupxi$ & \verb|\bfupxi| \\
omicron & $\omicron$ & \verb|\omicron| & $\bfomicron$ & \verb|\bfomicron| & $\upomicron$ & \verb|\upomicron| & $\bfupomicron$ & \verb|\bfupomicron| \\
pi & $\pi$ & \verb|\pi| & $\bfpi$ & \verb|\bfpi| & $\uppi$ & \verb|\uppi| & $\bfuppi$ & \verb|\bfuppi| \\
\kaishu{varpi} & $\varpi$ & \verb|\varpi| & $\bfvarpi$ & \verb|\bfvarpi| & $\upvarpi$ & \verb|\upvarpi| & $\bfupvarpi$ & \verb|\bfupvarpi| \\
rho & $\rho$ & \verb|\rho| & $\bfrho$ & \verb|\bfrho| & $\uprho$ & \verb|\uprho| & $\bfuprho$ & \verb|\bfuprho| \\
\kaishu{varrho} & $\varrho$ & \verb|\varrho| & $\bfvarrho$ & \verb|\bfvarrho| & $\upvarrho$ & \verb|\upvarrho| & $\bfupvarrho$ & \verb|\bfupvarrho| \\
sigma & $\sigma$ & \verb|\sigma| & $\bfsigma$ & \verb|\bfsigma| & $\upsigma$ & \verb|\upsigma| & $\bfupsigma$ & \verb|\bfupsigma| \\
\kaishu{varsigma} & $\varsigma$ & \verb|\varsigma| & $\bfvarsigma$ & \verb|\bfvarsigma| & $\upvarsigma$ & \verb|\upvarsigma| & $\bfupvarsigma$ & \verb|\bfupvarsigma| \\
tau & $\tau$ & \verb|\tau| & $\bftau$ & \verb|\bftau| & $\uptau$ & \verb|\uptau| & $\bfuptau$ & \verb|\bfuptau| \\
upsilon & $\upsilon$ & \verb|\upsilon| & $\bfupsilon$ & \verb|\bfupsilon| & $\upupsilon$ & \verb|\upupsilon| & $\bfupupsilon$ & \verb|\bfupupsilon| \\
phi & $\phi$ & \verb|\phi| & $\bfphi$ & \verb|\bfphi| & $\upphi$ & \verb|\upphi| & $\bfupphi$ & \verb|\bfupphi| \\
\kaishu{varphi} & $\varphi$ & \verb|\varphi| & $\bfvarphi$ & \verb|\bfvarphi| & $\upvarphi$ & \verb|\upvarphi| & $\bfupvarphi$ & \verb|\bfupvarphi| \\
chi & $\chi$ & \verb|\chi| & $\bfchi$ & \verb|\bfchi| & $\upchi$ & \verb|\upchi| & $\bfupchi$ & \verb|\bfupchi| \\
psi & $\psi$ & \verb|\psi| & $\bfpsi$ & \verb|\bfpsi| & $\uppsi$ & \verb|\uppsi| & $\bfuppsi$ & \verb|\bfuppsi| \\
omega & $\omega$ & \verb|\omega| & $\bfomega$ & \verb|\bfomega| & $\upomega$ & \verb|\upomega| & $\bfupomega$ & \verb|\bfupomega| \\}

\fsubsection{大写希腊字母——包括常用的10种字母}

其他14种与拉丁字母撞形,因而罕用。本模板使用XITS Math字体,Upsilon的字形$\Upsilon$与拉丁字母Y是有分别的,因此也提供了快捷命令支持,但表格中未列出。

\fbiaoge
{大写希腊字母命令}{tab:Greek}{9}{Y}
{名称 & \makecell[c]{默认\\正体} & \fangsong{命令} & \textbf{粗正体} & \fangsong{命令} & \textit{斜体} & \fangsong{命令} & \textbf{\textit{粗斜体}} & \fangsong{命令}}
{Gamma & $\Gamma$ & \verb|\Gamma| & $\bfGamma$ & \verb|\bfGamma| & $\itGamma$ & \verb|\itGamma| & $\bfitGamma$ & \verb|\bfitGamma| \\
Delta & $\Delta$ & \verb|\Delta| & $\bfDelta$ & \verb|\bfDelta| & $\itDelta$ & \verb|\itDelta| & $\bfitDelta$ & \verb|\bfitDelta| \\
Theta & $\Theta$ & \verb|\Theta| & $\bfTheta$ & \verb|\bfTheta| & $\itTheta$ & \verb|\itTheta| & $\bfitTheta$ & \verb|\bfitTheta| \\
Lambda & $\Lambda$ & \verb|\Lambda| & $\bfLambda$ & \verb|\bfLambda| & $\itLambda$ & \verb|\itLambda| & $\bfitLambda$ & \verb|\bfitLambda| \\
Xi & $\Xi$ & \verb|\Xi| & $\bfXi$ & \verb|\bfXi| & $\itXi$ & \verb|\itXi| & $\bfitXi$ & \verb|\bfitXi| \\
Pi & $\Pi$ & \verb|\Pi| & $\bfPi$ & \verb|\bfPi| & $\itPi$ & \verb|\itPi| & $\bfitPi$ & \verb|\bfitPi| \\
Sigma & $\Sigma$ & \verb|\Sigma| & $\bfSigma$ & \verb|\bfSigma| & $\itSigma$ & \verb|\itSigma| & $\bfitSigma$ & \verb|\bfitSigma| \\
Phi & $\Phi$ & \verb|\Phi| & $\bfPhi$ & \verb|\bfPhi| & $\itPhi$ & \verb|\itPhi| & $\bfitPhi$ & \verb|\bfitPhi| \\
Psi & $\Psi$ & \verb|\Psi| & $\bfPsi$ & \verb|\bfPsi| & $\itPsi$ & \verb|\itPsi| & $\bfitPsi$ & \verb|\bfitPsi| \\
Omega & $\Omega$ & \verb|\Omega| & $\bfOmega$ & \verb|\bfOmega| & $\itOmega$ & \verb|\itOmega| & $\bfitOmega$ & \verb|\bfitOmega| \\}
\biaozhu{注:为了兼容性考虑,大写希腊字母斜体也可以通过\texttt{var}前缀打出,且与\texttt{it}前缀的效果完全相同。如\texttt{\textbackslash itGamma}和\texttt{\textbackslash varGamma}都可以打出$\varGamma$;\texttt{\textbackslash bfitGamma}和\texttt{\textbackslash bfvarGamma}都可以打出$\bfvarGamma$。}

\fchapter{附录命令介绍}

使用\verb|\fchapter{}|命令在附录中开启新章节。

\fsection{附录中的Section}

使用\verb|\fsection{}|命令在附录中获得二级标题。

\fsubsection{附录中的Subsection}

使用\verb|\fsubsection{}|命令在附录中获得三级标题。

\fsubsubsection{附录中的Subsubsection}\label{appendix01:test}

使用\verb|\fsubsubsection{}|命令在附录中获得四级标题。

对标题的交叉引用方法不变,依旧是在需要引用时打上\verb|\label|,然后用\verb|\ref|引用,如:附录\ref{appendix01:test}。

\begin{enumerate}[kuohao]

    \item 插图要用\verb|ffigure|环境,用法不变:

        \begin{ffigure}[H]
        \centering
        \includegraphics[width=0.8\textwidth]{figures/附录图片.png}
        \caption{附录示例图片}
        \label{fig:app}
        \end{ffigure}\tuzhu{图注可以正常用。}

    \item 插表要用\verb|fbiaoge|或\verb|fsdbiaoge|环境,用法不变:

        \fbiaoge
        {附录表格示例}{tab:fnote}{2}{Y}
        { 东北大学 & 不愧是 }
        { 沈阳唯一 & 酒吧舞! \\}
        \biaozhu{表注也可以正常用。}

    \item 公式要用\verb|fequation|环境,用法不变:

        \begin{fequation}\label{eq:app}
            E_{生存}=\dfrac{(摸鱼)^{996}+划水^{内卷}\times(躺平+\sqrt[3]{佛系})}{\text{CPU}\times\text{KPI}\times 画饼}+典\times (绷+乐+蚌)^{孝}.
        \end{fequation}其中:
        \begin{itemize}
            \item $E_{生存}$:个体在当前社会环境下的生存能量。
            \item 摸鱼:在工作或学习时间进行非任务行为的效率。
            \item 996:经典“福报”工作时间模式,作为摸鱼效率的指数,表示在高压下摸鱼的边际效益递增。
            \item 划水:在集体项目中不出力的程度。
            \item 内卷:作为划水的指数,表示在内卷环境中,划水的风险与收益同步增加。
            \item 躺平:一种低欲望、不抵抗的生活态度,是能量的基础来源。
            \item 佛系:随缘、不强求的心态。开立方根表示过于佛系会导致能量增长缓慢。
            \item CPU/KPI/画饼:三者相乘,代表来自上级的精神控制(CPU)、绩效指标(KPI)和无法兑现的承诺(画饼),它们是消耗能量的主要分母。分母越大,生存能量越低。
            \item 典:作为一个系数,它放大了后面的情绪能量。
            \item 绷:指因离谱事件而即将破防的忍耐力。
            \item 乐:指纯粹的快乐和看乐子心态。
            \item 蚌:指最终无法忍受而破防大笑或崩溃。
            \item 孝:作为情绪能量的指数,指代“孝子贤孙”般的粉丝行为或对某些现象的无奈顺从,指数越高,情绪波动越大。

        \end{itemize}

         对附录中公式交叉引用,依旧是先给公式打上\verb|\label|,然后用\verb|\eqref|引用,方法不变,如:附录公式\eqref{eq:app}。
\end{enumerate}