\chapter{内容要求}

\section{论文题目}

毕业设计(论文)选题应结合本专业的(工程)实际问题,论文题目应以最恰当、简明的词语准确概括整个论文的核心内容,避免使用不常见的缩略词、缩写字。中文题目一般不宜超过24个字,必要时可增加副标题。外文题目一般不宜超过12个实词。

\section{摘要和关键词}

\subsection{中文摘要和中文关键词}

摘要内容应概括地反映出论文的主要内容,主要说明论文的研究目的、内容、方法、成果和结论。要突出论文的创新性成果,不要与引言相混淆。语言力求精练、准确。工科毕业论文摘要中不能全是定性描述,要对体现论文研究内容的关键性的数据指标做出说明。在摘要的下方另起一行,注明论文的关键词(3—5个)。

摘要撰写不能记流水账,单纯描述分别做了什么工作,或逐章内容罗列。要求说明本论文研究问题,使用关键方法,获得的结果分析及数据定量评价。

\subsection{英文摘要和英文关键词}

英文摘要内容与中文摘要相同。摘要后面注明英文关键词Key words(3—5个)。

\section{目录}

论文目录是论文的提纲,也是论文各章节组成部分的小标题。目录应按照章、节、条及次条四级标题编写,采用阿拉伯数字分级编号,要求标题层次清晰。只要按照本模板规范写作,目录即可正确生成,无需手动调整。

\section{正文}

正文是毕业论文的主体和核心部分,不同学科专业和不同的选题可以有不同的写作方式。正文一般包括以下几个方面:


\clearpage % 强制换页,让1.4.1的标题不孤立地待在页末,可以注释掉这行代码看效果

\subsection{引言或背景}\label{clearpage}

引言是论文正文的开端,引言应包括毕业论文选题的背景、目的和意义;对国内外研究现状和相关领域中已有的研究成果的简要评述;介绍本项研究工作研究设想、研究方法或实验设计、理论依据或实验基础;涉及范围和预期结果等。要求言简意赅,注意不要与摘要雷同或成为摘要的注解。不能大量摘抄教科书相关基础知识,不能介绍太多某领域的发展历史,而不写相关研究现状。

\subsection{主体}

论文主体是毕业论文的主要部分,必须言之成理,论据可靠,严格遵循本学科国际通行的学术规范。在写作上要注意结构合理、层次分明、重点突出,章节标题、公式图表符号必须规范统一。论文主体的内容根据不同学科有不同的特点,一般应包括以下几个方面:

\begin{enumerate}[kuohao]
  \item 毕业设计(论文)总体方案或选题的论证;
  \item 毕业设计(论文)各部分的设计实现,包括实验数据的获取、数据可行性及有效性的处理与分析、各部分的设计计算等;
  \item 对研究内容及成果的客观阐述,包括理论依据、创新见解、创造性成果及其改进与实际应用价值等;
  \item 论文主体的所有数据必须真实可靠,应推理正确、结论清晰;
  \item 硬件设计不能大量摘抄数据手册内容,不能只贴电路图等不进行必要的文字说明;
  \item 软件设计不能大量粘贴源代码,不能只贴流程图,不加必要的文字说明;
  \item 实验数据分析不能大量粘贴条件类似,同质化的原始数据,不分析总结。不能全是定性分析结论,必须有定量指标分析。
\end{enumerate}

\subsection{结论}

结论是毕业论文的总结,是整篇论文的归宿。应精炼、准确、完整。着重阐述自己的创造性成果及其在本研究领域中的意义、作用,还可进一步提出需要讨论的问题和建议。

结论撰写不能记流水账,单纯描述分别做了什么工作,或逐章内容罗列。要求对全文研究内容和方法等做出凝练总结。建议按照(1)、(2)条目列出。

\section{中外文参考文献}

毕业论文的撰写应本着严谨求实的科学态度,凡有引用他人成果之处,均应按论文中所引用的顺序列于文末,并且所有参考文献必须在正文中有引用标注。参考文献的著录均应符合国家有关标准(按照GB7714—2005执行)。一篇论著在论文中多处引用时,在参考文献中只应出现一次,序号以第一次出现的位置为准。参考文献至少40篇。

\section{附录}

附录应包括一篇外文文献的中文译文及其原文,对于一些不宜放在正文中的重要支撑材料,也可编入毕业论文的附录中,包括某些重要的原始数据、详细数学推导、程序全文及其说明、复杂的图表、设计图纸等一系列需要补充提供的说明材料。

\section{致谢}

表达作者对完成论文和学业提供帮助的老师、同学、领导、同事及亲属的感激之情。致谢书写语气不能太随意,要严肃大方。