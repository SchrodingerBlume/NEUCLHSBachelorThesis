% \iffalse meta-comment
% !TeX program = XeLaTeX
% !TeX encoding = UTF-8
%
%<*internal>
\iffalse
%</internal>
%
%<*readme>
----------------------------------------------------------------
NEUCLHSBachelorThesis --- Bachelor Thesis LaTeX Template for
                          College of Life and Health Sciences,
                          Northeastern University (NEU)

Author:  Wu Junhao (Schrödinger's Blume)
E-mail:  neuclhswjh@163.com
         junhaowu_hit@163.com
License: LPPL v1.3c or later
Version: v1.1.1
Date:    2025-11-17

This package provides a LaTeX document class for bachelor thesis
of College of Life and Health Sciences, Northeastern University.

INSTALLATION
============

Run:
    xelatex NEUCLHSBachelorThesis.dtx

to generate the installation script, then run:

    latex NEUCLHSBachelorThesis.ins

to generate the class and package files.

DOCUMENTATION
=============

To generate the documentation, run:

    xelatex NEUCLHSBachelorThesis.dtx
    makeindex -s gind.ist NEUCLHSBachelorThesis.idx
    makeindex -s gglo.ist -o NEUCLHSBachelorThesis.gls NEUCLHSBachelorThesis.glo
    xelatex NEUCLHSBachelorThesis.dtx
    xelatex NEUCLHSBachelorThesis.dtx

%</readme>
%
%<*internal>
\fi
\def\nameofplainTeX{plain}
\ifx\fmtname\nameofplainTeX\else
  \expandafter\begingroup
\fi
%</internal>
%
%<*install>
\input docstrip.tex
\keepsilent
\askforoverwritefalse

\preamble
----------------------------------------------------------------
NEUCLHSBachelorThesis --- Bachelor Thesis LaTeX Template for
                          College of Life and Health Sciences,
                          Northeastern University (NEU)

Author:  Wu Junhao (Schrödinger's Blume)
E-mail:  neuclhswjh@163.com
         junhaowu_hit@163.com
License: LPPL v1.3c or later
Version: v1.1.1
Date:    2025-11-17
----------------------------------------------------------------
\endpreamble

\postamble

Copyright (C) 2025 by Wu Junhao <neuclhswjh@163.com>

This work may be distributed and/or modified under the
conditions of the LaTeX Project Public License (LPPL), either
version 1.3c of this license or (at your option) any later
version. The latest version of this license is in the file:

   http://www.latex-project.org/lppl.txt

This work has the LPPL maintenance status `maintained'.

The Current Maintainer of this work is Wu Junhao.

This work consists of the files NEUCLHSBachelorThesis.dtx,
NEUCLHSBachelorThesis.ins and the derived files
NEUCLHSBachelorThesis.cls and untilchapter.sty.

\endpostamble

\usedir{tex/latex/neuclhsbachelorthesis}
\generate{
  \file{NEUCLHSBachelorThesis.cls}{\from{\jobname.dtx}{class}}
  \file{untilchapter.sty}{\from{\jobname.dtx}{package}}
}

%</install>
%<install>\endbatchfile
%
%<*internal>
\usedir{source/latex/neuclhsbachelorthesis}
\generate{
  \file{\jobname.ins}{\from{\jobname.dtx}{install}}
}
\nopreamble\nopostamble
\usedir{doc/latex/neuclhsbachelorthesis}
\generate{
  \file{README.txt}{\from{\jobname.dtx}{readme}}
}
\ifx\fmtname\nameofplainTeX
  \expandafter\endbatchfile
\else
  \expandafter\endgroup
\fi
%</internal>
%
% \fi
%
% \iffalse
%<*driver>
\ProvidesFile{NEUCLHSBachelorThesis.dtx}
%</driver>
%<class>\NeedsTeXFormat{LaTeX2e}[2020/10/01]
%<class>\ProvidesClass{NEUCLHSBachelorThesis}
%<class>  [2025/11/17 v1.1.1 Bachelor Thesis Template for NEU CLHS]
%<package>\NeedsTeXFormat{LaTeX2e}[2020/10/01]
%<package>\ProvidesPackage{untilchapter}
%<package>  [2025/11/17 v1.1.1 Page style management for NEU thesis]
%
%<*driver>
\documentclass[a4paper]{ltxdoc}
\usepackage{xeCJK}
\usepackage{hyperref}
\setCJKmainfont{SimSun}[AutoFakeBold=2.5]
\setCJKsansfont{SimHei}
\setCJKmonofont{FangSong}
\hypersetup{
  colorlinks=true,
  linkcolor=blue,
  urlcolor=blue,
  pdfauthor={Wu Junhao},
  pdftitle={NEUCLHSBachelorThesis Documentation}
}
\EnableCrossrefs
\CodelineIndex
\RecordChanges
\begin{document}
  \DocInput{\jobname.dtx}
  \PrintChanges
  \PrintIndex
\end{document}
%</driver>
% \fi
%
% \GetFileInfo{\jobname.dtx}
% \DoNotIndex{\newcommand,\renewcommand,\providecommand}
% \DoNotIndex{\RequirePackage,\LoadClass,\DeclareOption}
% \DoNotIndex{\ProcessOptions,\relax,\MessageBreak}
%
% \title{^^A
%   \textsf{NEUCLHSBachelorThesis} --- ^^A
%   东北大学生命科学与健康学院\\本科毕业论文模板\thanks{^^A
%     This file describes version \fileversion, last revised \filedate.^^A
%   }^^A
% }
%
% \author{^^A
%   吴俊豪 (Schrödinger's Blume)\thanks{^^A
%     E-mail: \texttt{neuclhswjh@163.com}, \texttt{junhaowu\_hit@163.com}^^A
%   }^^A
% }
%
% \date{Released \filedate}
%
% \maketitle
%
% \changes{v1.0.0}{2024/03/15}{首次发布}
% \changes{v1.1.0}{2025/10/15}{结构优化,配置文件模块化}
% \changes{v1.1.1}{2025/11/17}{修正细节问题,优化数学排版}
%
% \begin{abstract}
% \noindent
% 本模板为东北大学生命科学与健康学院本科毕业论文 \LaTeX{} 模板。
% 这是一个\textbf{非官方}但严格按照学院要求制作的文档类。
% 模板支持中英文混排、数学公式、参考文献管理(GB/T~7714--2025标准)等功能,
% 提供彩色封面、声明页、中英文摘要、目录、正文、参考文献、致谢、附录等完整结构。
% \end{abstract}
%
% \tableofcontents
%
% \section{简介}
%
% \subsection{项目背景}
%
% 本项目为东北大学生命科学与健康学院本科毕业设计(论文)\LaTeX{} 模板。
% 相比传统的 Word 排版,\LaTeX{} 具有以下优势:
%
% \begin{itemize}
%   \item \textbf{排版专业}:学术界公认的专业排版系统,输出质量远超普通文字处理软件
%   \item \textbf{公式美观}:数学公式渲染效果优秀,特别适合理工科论文
%   \item \textbf{引用管理}:自动管理参考文献、图表编号、交叉引用等
%   \item \textbf{专注内容}:将内容与格式分离,让你专注于论文内容本身
% \end{itemize}
%
% \subsection{适用对象}
%
% \begin{itemize}
%   \item 东北大学生命科学与健康学院本科毕业生
%   \item 希望使用 \LaTeX{} 撰写毕业论文的同学
%   \item 对 Word 模板感到困扰的同学
% \end{itemize}
%
% \subsection{系统要求}
%
% \begin{description}
%   \item[编译引擎] 必须使用 \XeLaTeX{}(不支持 pdf\LaTeX{})
%   \item[\TeX{} 发行版] 推荐 \TeX{} Live 2024 或更新版本,或 Mac\TeX{}
%   \item[中文字体] 需要中易宋体(SimSun)和中易黑体(SimHei)
%   \begin{itemize}
%     \item Windows 用户:系统自带所需字体
%     \item macOS/Linux 用户:需要手动安装或将字体文件放入 \texttt{fonts/} 目录
%   \end{itemize}
% \end{description}
%
% \subsection{主要特性}
%
% \begin{itemize}
%   \item 格式规范:严格按照东北大学生命科学与健康学院本科毕业论文格式要求制作
%   \item 开箱即用:配置简单,修改几个参数即可开始撰写
%   \item 模块化设计:所有配置文件独立管理(位于 \texttt{settings/} 目录),结构清晰
%   \item 中文支持:基于 \textsf{ctexbook} 类,完善的中文排版支持
%   \item 参考文献:支持国标 GB/T~7714--2025 参考文献格式(使用 Bib\LaTeX{})
%   \item 双面打印:支持单面/双面打印模式,对称/固定页边距可选
%   \item 精美封面:内置彩色封面和封底(PDF格式)
%   \item 数学支持:使用 XITS Math 字体,支持 unicode-math 宏包
% \end{itemize}
%
% \subsection{许可证}
%
% 本模板采用 \LaTeX{} Project Public License (LPPL) v1.3c 或更高版本发布。
% 您可以自由使用、修改和分发本模板,但需要遵守 LPPL 协议的条款。
%
% \section{使用说明}
%
% \subsection{基本用法}
%
% 在您的主文档(如 \texttt{main.tex})中使用本文档类:
%
% \begin{verbatim}
% \documentclass{NEUCLHSBachelorThesis}
% 
% % 加载配置文件
% % =========================== %
% ========  封面设置  ======== %
% =========================== %

% ========= 信息设置 ========= %
\SetCoverTitle{%
东北大学生命科学与健康学院本科毕业论文\LaTeX 模板\coverxt{V-1.0.0}\\
——一份使用示例文档
}
% 标题会自动换行,如对断行位置不满意,可以用\\强制换行
% 在封面标题中使用斜体需要使用\coverxt命令

% \SetCoverCollege{}可以修改学院

\SetCoverID{您的学号}

\SetCoverMajor{您的专业}

\SetCoverName{您的姓名}

\SetCoverTeacher{教师姓名}

\SetCoverTeacherTitle{职称}

\SetCoverYear{2026} % 年份,输入阿拉伯数字

\SetCoverMonth{6} % 月份,输入阿拉伯数字

% ========= 大封面设置 ========= %
\usebigcovertrue % 启用蓝色大封面,注释掉即关闭
% % =========================== %
% ========  打印设置  ======== %
% =========================== %

% 打印模式可选参数:
% a. twoside:双面打印,会生成空白页
% b. oneside:单面打印,不会生成空白页
% 无论哪种,新的节都会被放置在奇数页
\def\PrintMode{twoside}

% 对称页边距可选开关:
% a. true:对称页边距
% b. false:固定页边距
\def\SymmetricMargins{false}

% 建议参数:
% a. 自己阅读查看的电子版:oneside+false
% b. 最终提交的电子版本:twoside+false
% c. 打印版本:twoside+true
% d. 折磨自己或他人:oneside+true
% 
% \begin{document}
% % 文档内容...
% \end{document}
% \end{verbatim}
%
% \subsection{配置文件}
%
% 本模板通过独立的配置文件进行设置,主要配置文件位于 \texttt{settings/} 目录:
%
% \begin{description}
%   \item[\texttt{cover.tex}] \textbf{必需}。封面信息(论文标题、作者、学号、专业、指导教师等)
%   \item[\texttt{statement.tex}] \textbf{必需}。声明页信息(签名图片、日期)
%   \item[\texttt{printmode.tex}] \textbf{必需}。打印模式(\texttt{twoside}/\texttt{oneside},对称/固定页边距)
%   \item[\texttt{chapter.tex}] 可选。控制包含哪些章节
%   \item[\texttt{appendix.tex}] 可选。控制包含哪些附录
%   \item[\texttt{typeset.tex}] 可选。调整字距、行距等排版细节
%   \item[\texttt{bibliography.tex}] 可选。参考文献排版设置
%   \item[\texttt{hyperref.tex}] 可选。超链接颜色和样式
% \end{description}
%
% \subsection{编译方法}
%
% \subsubsection{完整编译流程}
%
% 首次编译或修改参考文献后,需要完整编译:
%
% \begin{verbatim}
% xelatex main
% biber main
% xelatex main
% xelatex main
% \end{verbatim}
%
% \subsubsection{日常编译}
%
% 如果只修改了正文内容,只需:
%
% \begin{verbatim}
% xelatex main
% \end{verbatim}
%
% \subsubsection{使用 latexmk}
%
% 推荐使用 \texttt{latexmk} 自动处理编译流程:
%
% \begin{verbatim}
% latexmk -xelatex main.tex
% \end{verbatim}
%
% \subsection{文档结构}
%
% 论文内容文件位于 \texttt{data/} 目录:
%
% \begin{itemize}
%   \item \texttt{cabstract.tex} --- 中文摘要
%   \item \texttt{eabstract.tex} --- 英文摘要
%   \item \texttt{chap01.tex} --- 第一章
%   \item \texttt{chap02.tex} --- 第二章
%   \item \texttt{ack.tex} --- 致谢
%   \item \texttt{appendix01.tex} --- 附录1
%   \item \texttt{references.bib} --- 参考文献数据库
% \end{itemize}
%
% \StopEventually{^^A
%   \PrintChanges
%   \PrintIndex
% }
%
% \section{代码实现}
%
% \subsection{NEUCLHSBachelorThesis 文档类}
%
% 下面是 \textsf{NEUCLHSBachelorThesis} 文档类的完整实现。
%
% \iffalse
%<*class>
% \fi
%    \begin{macrocode}
\ProvidesClass{NEUCLHSBachelorThesis}
\providecommand{\Version}{V-1.1.1}
\providecommand{\PrintMode}{twoside}
\LoadClass[UTF8,\PrintMode,a4paper,12pt]{ctexbook}
\RequirePackage{xeCJK}
\RequirePackage{pdfpages}
\RequirePackage{float}
\RequirePackage{flafter} % 不让浮动体出现在调用之前
\RequirePackage{subcaption}
\captionsetup{hypcap=true}
\RequirePackage{xcolor}
\RequirePackage{ragged2e}
\RequirePackage{calc}
\RequirePackage{lipsum}
\RequirePackage{comment}
\RequirePackage{tikz}
\RequirePackage{needspace}

% 定义手动换页命令\huanye,根据oneside/twoside决定启用与否
\if@twoside
  \newcommand{\huanye}{\clearpage\thispagestyle{empty}\ }
\else
  \newcommand{\huanye}{}
\fi

% 单面打印时,确保章节从奇数页开始
\if@twoside
  % twoside,使用默认的openright行为
\else
  % oneside,在\chapter前检查页码
  \let\oldchapter\chapter
  \renewcommand{\chapter}{%
    \clearpage
    \ifodd\value{page}%
      % 当前是奇数页,章节将出现在奇数页,不需要调整
    \else
      % 当前是偶数页,章节将出现在偶数页,需要跳过一页
      \stepcounter{page}%
    \fi
    \oldchapter
  }
\fi

\RequirePackage{emptypage}

\raggedbottom % 取消齐底对齐

%------- 字体 --------%

%--- 西文字体配置 ---%
\setmainfont{Times New Roman}

%--- 数学字体配置 ---%
\xeCJKsetup{%
CJKmath=true, % 公式中直接写中文
}

\RequirePackage[partial=upright]{unicode-math}
\RequirePackage{bm} % for \mathbfit

\newfontfamily\timesrm{Times New Roman}[
  BoldFont = Times New Roman,
  BoldFeatures = {FakeBold=0}
]

% StylisticSet=8使积分号直立
\setmathfont{XITS Math}[StylisticSet=8]

\RequirePackage{upgreek}
\providecommand{\dd}{\mathop{}\!\mathrm{d}}
\providecommand{\ee}{\mathrm{e}}
\providecommand{\ii}{\mathrm{i}}
\providecommand{\jj}{\mathrm{j}}

\renewcommand\mathellipsis{\cdots} % 省略号居中

\AtBeginDocument{
\renewcommand{\Re}{\operatorname{Re}} % 罗马体\Re
\renewcommand{\Im}{\operatorname{Im}} % 罗马体\Im
}

% \parallel用宋体,使之倾斜
\IfFileExists{./fonts/SIMSUN.TTC}{
    \setmathfont{SIMSUN.TTC}[
        Path=./fonts/,
        Extension=.TTC,
        range={"2225}
    ]
}{
    \IfFontExistsTF{SimSun}{
        \setmathfont{SimSun}[range={"2225}]
    }{
        \IfFontExistsTF{FandolSong-Regular}{
        \setmathfont{FandolSong-Regular}[range={"2225}]}
{% 回退默认
}}}

% 小写希腊字母粗体命令
\newcommand{\bfalpha}{\symbf{\alpha}}
\newcommand{\bfbeta}{\symbf{\beta}}
\newcommand{\bfgamma}{\symbf{\gamma}}
\newcommand{\bfdelta}{\symbf{\delta}}
\newcommand{\bfepsilon}{\symbf{\epsilon}}
\newcommand{\bfvarepsilon}{\symbf{\varepsilon}}
\newcommand{\bfzeta}{\symbf{\zeta}}
\newcommand{\bfeta}{\symbf{\eta}}
\newcommand{\bftheta}{\symbf{\theta}}
\newcommand{\bfvartheta}{\symbf{\vartheta}}
\newcommand{\bfiota}{\symbf{\iota}}
\newcommand{\bfkappa}{\symbf{\kappa}}
\newcommand{\bfvarkappa}{\symbf{\varkappa}}
\newcommand{\bflambda}{\symbf{\lambda}}
\newcommand{\bfmu}{\symbf{\mu}}
\newcommand{\bfnu}{\symbf{\nu}}
\newcommand{\bfxi}{\symbf{\xi}}
\newcommand{\bfomicron}{\symbf{\omicron}}
\newcommand{\bfpi}{\symbf{\pi}}
\newcommand{\bfvarpi}{\symbf{\varpi}}
\newcommand{\bfrho}{\symbf{\rho}}
\newcommand{\bfvarrho}{\symbf{\varrho}}
\newcommand{\bfsigma}{\symbf{\sigma}}
\newcommand{\bfvarsigma}{\symbf{\varsigma}}
\newcommand{\bftau}{\symbf{\tau}}
\newcommand{\bfupsilon}{\symbf{\upsilon}}
\newcommand{\bfphi}{\symbf{\phi}}
\newcommand{\bfvarphi}{\symbf{\varphi}}
\newcommand{\bfchi}{\symbf{\chi}}
\newcommand{\bfpsi}{\symbf{\psi}}
\newcommand{\bfomega}{\symbf{\omega}}

% 小写希腊字母粗体正体命令
\newcommand{\bfupalpha}{\symbfup{\alpha}}
\newcommand{\bfupbeta}{\symbfup{\beta}}
\newcommand{\bfupgamma}{\symbfup{\gamma}}
\newcommand{\bfupdelta}{\symbfup{\delta}}
\newcommand{\bfupepsilon}{\symbfup{\epsilon}}
\newcommand{\bfupvarepsilon}{\symbfup{\varepsilon}}
\newcommand{\bfupzeta}{\symbfup{\zeta}}
\newcommand{\bfupeta}{\symbfup{\eta}}
\newcommand{\bfuptheta}{\symbfup{\theta}}
\newcommand{\bfupvartheta}{\symbfup{\vartheta}}
\newcommand{\bfupiota}{\symbfup{\iota}}
\newcommand{\bfupkappa}{\symbfup{\kappa}}
\newcommand{\bfupvarkappa}{\symbfup{\varkappa}}
\newcommand{\bfuplambda}{\symbfup{\lambda}}
\newcommand{\bfupmu}{\symbfup{\mu}}
\newcommand{\bfupnu}{\symbfup{\nu}}
\newcommand{\bfupxi}{\symbfup{\xi}}
\newcommand{\bfupomicron}{\symbfup{\omicron}}
\newcommand{\bfuppi}{\symbfup{\pi}}
\newcommand{\bfupvarpi}{\symbfup{\varpi}}
\newcommand{\bfuprho}{\symbfup{\rho}}
\newcommand{\bfupvarrho}{\symbfup{\varrho}}
\newcommand{\bfupsigma}{\symbfup{\sigma}}
\newcommand{\bfupvarsigma}{\symbfup{\varsigma}}
\newcommand{\bfuptau}{\symbfup{\tau}}
\newcommand{\bfupupsilon}{\symbfup{\upsilon}}
\newcommand{\bfupphi}{\symbfup{\phi}}
\newcommand{\bfupvarphi}{\symbfup{\varphi}}
\newcommand{\bfupchi}{\symbfup{\chi}}
\newcommand{\bfuppsi}{\symbfup{\psi}}
\newcommand{\bfupomega}{\symbfup{\omega}}

% 大写希腊字母粗体命令
\newcommand{\bfGamma}{\symbf{\Gamma}}
\newcommand{\bfDelta}{\symbf{\Delta}}
\newcommand{\bfTheta}{\symbf{\Theta}}
\newcommand{\bfLambda}{\symbf{\Lambda}}
\newcommand{\bfXi}{\symbf{\Xi}}
\newcommand{\bfPi}{\symbf{\Pi}}
\newcommand{\bfSigma}{\symbf{\Sigma}}
\newcommand{\bfUpsilon}{\symbf{\Upsilon}}
\newcommand{\bfPhi}{\symbf{\Phi}}
\newcommand{\bfPsi}{\symbf{\Psi}}
\newcommand{\bfOmega}{\symbf{\Omega}}

% 大写希腊字母粗体斜体命令
\newcommand{\bfitGamma}{\symbfit{\Gamma}}
\newcommand{\bfitDelta}{\symbfit{\Delta}}
\newcommand{\bfitTheta}{\symbfit{\Theta}}
\newcommand{\bfitLambda}{\symbfit{\Lambda}}
\newcommand{\bfitXi}{\symbfit{\Xi}}
\newcommand{\bfitPi}{\symbfit{\Pi}}
\newcommand{\bfitSigma}{\symbfit{\Sigma}}
\newcommand{\bfitUpsilon}{\symbfit{\Upsilon}}
\newcommand{\bfitPhi}{\symbfit{\Phi}}
\newcommand{\bfitPsi}{\symbfit{\Psi}}
\newcommand{\bfitOmega}{\symbfit{\Omega}}

% 大写希腊字母var版粗体斜体命令
\newcommand{\bfvarGamma}{\symbfit{\Gamma}}
\newcommand{\bfvarDelta}{\symbfit{\Delta}}
\newcommand{\bfvarTheta}{\symbfit{\Theta}}
\newcommand{\bfvarLambda}{\symbfit{\Lambda}}
\newcommand{\bfvarXi}{\symbfit{\Xi}}
\newcommand{\bfvarPi}{\symbfit{\Pi}}
\newcommand{\bfvarSigma}{\symbfit{\Sigma}}
\newcommand{\bfvarUpsilon}{\symbfit{\Upsilon}}
\newcommand{\bfvarPhi}{\symbfit{\Phi}}
\newcommand{\bfvarPsi}{\symbfit{\Psi}}
\newcommand{\bfvarOmega}{\symbfit{\Omega}}

\renewcommand{\varGamma}{\itGamma}
\renewcommand{\varDelta}{\itDelta}
\AtBeginDocument{\let\varTheta\itTheta}
\renewcommand{\varLambda}{\itLambda}
\renewcommand{\varXi}{\itXi}
\renewcommand{\varPi}{\itPi}
\renewcommand{\varSigma}{\itSigma}
\renewcommand{\varGamma}{\itGamma}
\renewcommand{\varUpsilon}{\itUpsilon}
\renewcommand{\varPhi}{\itPhi}
\renewcommand{\varPsi}{\itPsi}
\renewcommand{\varOmega}{\itOmega}

% up和bfup
\setmathfont{Times New Roman}[
  range = {up/{latin,Latin}},
]
\IfFontExistsTF{Times New Roman Bold}{
  \setmathfont{Times New Roman Bold}[
    range = {bfup/{latin,Latin}},
  ]
}{
  % 原生粗体缺失就用伪粗
  \setmathfont{Times New Roman}[
    range    = {bfup/{latin,Latin}},
    FakeBold = 2.0,
  ]
}
% it
\IfFontExistsTF{Times New Roman Italic}{
  \setmathfont{Times New Roman Italic}[
    range = {it/{latin,Latin}},
  ]
}{
  \setmathfont{Times New Roman}[
    range     = {it/{latin,Latin}},
    FakeSlant = 0.33,
  ]
}
% bfit
\IfFontExistsTF{Times New Roman Bold Italic}{
  \setmathfont{Times New Roman Bold Italic}[
    range = {bfit/{latin,Latin}},
  ]
}{
  \IfFontExistsTF{Times New Roman Bold}{
    % 粗体+伪斜
    \setmathfont{Times New Roman Bold}[
      range     = {bfit/{latin,Latin}},
      FakeSlant = 0.33,
    ]
  }{
    % 伪粗+伪斜
    \setmathfont{Times New Roman}[
      range     = {bfit/{latin,Latin}},
      FakeBold  = 2.0,
      FakeSlant = 0.33,
    ]
  }
}

\setmathfont{Times New Roman}[range={"03B5, "03C2, "03C6, "03B1, "03B2, "03B3, "03B4, "03B5, "03B6, "03B7, "03B8, "03B9, "03BA, "03BB, "03BC, "03BD, "03BE, "03BF, "03C0, "03C1, "03C3, "03C4, "03C5, "03C6, "03C7, "03C8, "03C9}]%

% 板粗体(空心体)和花体
% 参考https://www.crexyer.com/2019/06/mathbb-show-different-chars-with-unicode-math-package/
\DeclareMathAlphabet{\mathcal}{OMS}{cmsy}{m}{n} % 这个方案采用克努斯于1977–1979年设计的样式,不支持小写花体,参考https://logic.pku.edu.cn/docs/20211226234023970182.pdf
% 注释掉上一条命令,启用下一行,就可以支持小写花体,但我不喜欢这个字体
% \setmathfont[range={\mathcal}]{TeXGyrePagellaMath-Regular}
\let\mathbb\relax % remove the definition by unicode-math
\DeclareMathAlphabet{\mathbb}{U}{msb}{m}{n}

\RequirePackage{newunicodechar}

% 让unicode的℃符号能显示
\newunicodechar{℃}{\unit{\celsius}}

% 让unicode的罗马数字能显示
\newunicodechar{Ⅰ}{I}
\newunicodechar{Ⅱ}{II}
\newunicodechar{Ⅲ}{III}
\newunicodechar{Ⅳ}{IV}
\newunicodechar{Ⅴ}{V}
\newunicodechar{Ⅵ}{VI}
\newunicodechar{Ⅶ}{VII}
\newunicodechar{Ⅷ}{VIII}
\newunicodechar{Ⅸ}{IX}
\newunicodechar{Ⅹ}{X}
\newunicodechar{Ⅺ}{XI}
\newunicodechar{Ⅻ}{XII}

\newunicodechar{·}{$\cdot$}

%--- SI单位 ---%
\RequirePackage{siunitx}
\sisetup{
  inter-unit-product = \cdot, % 用 \cdot 分隔
  list-separator = {、},
  list-final-separator = {和},
  list-pair-separator = {和},
  list-units = single,
  range-units = single,
  range-phrase = \blx,
  separate-uncertainty = true % 不确定度用±
}

%--- 中文字体配置 ---%

% 在非Windows平台(如Overleaf或Linux平台)编译时,请自行将Windows下默认宋黑仿楷字体(中易字库)“SIMSUN.TTC、SIMHEI.TTF、SIMFANG.TTF及SIMKAI.TTF”复制到fonts文件夹下

% 宋体
\IfFileExists{./fonts/SIMSUN.TTC}{
    % 本地字体存在,使用本地字体
    \setCJKmainfont[
        Path=./fonts/,
        Extension=.TTC,
        AutoFakeBold=2.17,
        AutoFakeSlant=0.333,
        UprightFont=SIMSUN
    ]{simsun}
}{
    % 本地字体不存在,尝试系统字体
    \IfFontExistsTF{SimSun}{
        \setCJKmainfont{SimSun}[
            AutoFakeBold=2.17,
            AutoFakeSlant=0.333
        ]
    }{
        % 系统字体不存在,回退到Fandol宋体
        \setCJKmainfont{FandolSong-Regular}[
            AutoFakeBold=2.17,
            AutoFakeSlant=0.333
        ]
    }
}

\IfFileExists{./fonts/SIMSUN.TTC}{
    \setCJKfamilyfont{song_windows}[
        Path=./fonts/,
        Extension=.TTC,
        AutoFakeBold=2.17,
        AutoFakeSlant=0.333,
        UprightFont=SIMSUN
    ]{simsun}
}{
    \IfFontExistsTF{SimSun}{
        \setCJKfamilyfont{song_windows}{SimSun}[
            AutoFakeBold=2.17,
            AutoFakeSlant=0.333
        ]
    }{
        \setCJKfamilyfont{song_windows}{FandolSong-Regular}[
            AutoFakeBold=2.17,
            AutoFakeSlant=0.333
        ]
    }
}
\renewcommand{\songti}{\CJKfamily{song_windows}}

% 西文宋体
\IfFileExists{./fonts/SIMSUN.TTC}{
    \newfontfamily\SimSunWestern{SIMSUN}[
        Path=./fonts/,
        Extension=.TTC,
        UprightFont=*
    ]
}{
    \IfFontExistsTF{SimSun}{
        \newfontfamily\SimSunWestern{SimSun}
    }{
        \newfontfamily\SimSunWestern{FandolSong-Regular}
    }
}
\newcommand{\songtiarabic}[1]{{\SimSunWestern #1}}

% 黑体
\IfFileExists{./fonts/SIMHEI.TTF}{
    \setCJKfamilyfont{hei_windows}[
        Path=./fonts/,
        Extension=.TTF,
        AutoFakeSlant=0.333,
        UprightFont=SIMHEI
    ]{simhei}
}{
    \IfFontExistsTF{SimHei}{
        \setCJKfamilyfont{hei_windows}{SimHei}[
            AutoFakeSlant=0.333
        ]
    }{
        \setCJKfamilyfont{hei_windows}{FandolHei-Regular}[
            AutoFakeSlant=0.333
        ]
    }
}
\renewcommand{\heiti}{
  \CJKfamily{hei_windows}%
  \fontspec{Times New Roman}
}

% 西文黑体
\IfFileExists{./fonts/SIMHEI.TTF}{
    \newfontfamily\SimHeiWestern{SIMHEI}[
        Path=./fonts/,
        Extension=.TTF,
        UprightFont=*
    ]
}{
    \IfFontExistsTF{SimHei}{
        \newfontfamily\SimHeiWestern{SimHei}
    }{
        \newfontfamily\SimHeiWestern{FandolHei-Regular}
    }
}
\newcommand{\heitiarabic}[1]{{\SimHeiWestern #1}}

% 仿宋
\IfFileExists{./fonts/SIMFANG.TTF}{
    \setCJKfamilyfont{fang_windows}[
        Path=./fonts/,
        Extension=.TTF,
        AutoFakeBold=2.17,
        AutoFakeSlant=0.333,
        UprightFont=SIMFANG
    ]{simfang}
}{
    \IfFontExistsTF{FangSong}{
        \setCJKfamilyfont{fang_windows}{FangSong}[
            AutoFakeBold=2.17,
            AutoFakeSlant=0.333
        ]
    }{
        \setCJKfamilyfont{fang_windows}{FandolFang-Regular}[
            AutoFakeBold=2.17,
            AutoFakeSlant=0.333
        ]
    }
}
\renewcommand{\fangsong}{\CJKfamily{fang_windows}}

% 楷体
\IfFileExists{./fonts/SIMKAI.TTF}{
    \setCJKfamilyfont{kai_windows}[
        Path=./fonts/,
        Extension=.TTF,
        AutoFakeBold=2.17,
        AutoFakeSlant=0.333,
        UprightFont=SIMKAI
    ]{simkai}
}{
    \IfFontExistsTF{KaiTi}{
        \setCJKfamilyfont{kai_windows}{KaiTi}[
            AutoFakeBold=2.17,
            AutoFakeSlant=0.333
        ]
    }{
        \setCJKfamilyfont{kai_windows}{FandolKai-Regular}[
            AutoFakeBold=2.17,
            AutoFakeSlant=0.333
        ]
    }
}
\renewcommand{\kaishu}{\CJKfamily{kai_windows}}

%--- 字号 ---%

% 这几个命令是我排版用的,千万不要动
% 如果有用到其他字号的需要,建议还是用原生的\zihao{}吧
\newcommand{\xiaoer}{\fontsize{18.06749pt}{23.086043pt}\selectfont}
\newcommand{\sihao}{\fontsize{14.05249pt}{23.086043pt}\selectfont}
\newcommand{\xiaosi}{\fontsize{12.045pt}{23.086043pt}\selectfont}
\newcommand{\wuhao}{\fontsize{10.53937pt}{10.53937pt}\selectfont}
\newcommand{\xiaowu}{\fontsize{9.03374pt}{12.5pt}\selectfont}

%--- 快捷符号 ---%
\newcommand{\blx}{\symbol{"FF5E}} % 一字宽波浪线,符合国标,U+FF5E
%\newcommand{\blx}{\textasciitilde} % 半宽波浪线
\newcommand{\slh}{……} % 省略号
\newcommand{\ssd}{\kern-0.15em\unit{\celsius}} % 摄氏度

% -- 上下标 --%
% 修改上下标大小和高度
\let\oldtextsuperscript\textsuperscript
\let\oldtextsubscript\textsubscript
\renewcommand{\textsuperscript}[1]{%
  \oldtextsuperscript{\raisebox{-0.2ex}{\scalebox{0.85}{#1}}}%
}
\renewcommand{\textsubscript}[1]{%
  \oldtextsubscript{\raisebox{0.55ex}{\scalebox{0.85}{#1}}}%
}

% 定义更适合中国宝宝的上下标
\newcommand{\shang}[1]{\textsuperscript{#1}}
\newcommand{\xia}[1]{\textsubscript{#1}}

\RequirePackage{xspace}
\RequirePackage{hologo} % TeX图标
\newcommand{\subib}[1]{%
  {\fontspec{Times New Roman}\MakeUppercase{\scalebox{1.2}[1.2]{\oldtextsubscript{\raisebox{0.325ex}{\hskip -0.1em #1\hskip -0.175em}}}}}%
}
\newcommand{\Biblogo}{B\subib{IB}}
\newcommand{\BibLaTeX}{\Biblogo\LaTeX}

%------- 页面设置 -------%
\RequirePackage{geometry}
\makeatletter
\def\@true{true}
\ifx\SymmetricMargins\@true
    % 对称页边距模式(印刷版)
    \geometry{
        inner=30mm,
        outer=25mm,
        top=27.3mm,
        bottom=28mm,
        headsep=7.1mm,
        footskip=8.6mm
    }
\else
    % 固定页边距模式(电子版)
    \geometry{
        left=30mm,
        right=25mm,
        top=27.3mm,
        bottom=28mm,
        headsep=7.1mm,
        footskip=8.6mm
    }
    % 强制奇偶页使用相同的左右边距
    \setlength{\oddsidemargin}{\dimexpr 30mm - 1in\relax}
    \setlength{\evensidemargin}{\dimexpr 30mm - 1in\relax}
\fi
\makeatother

\setlength{\headheight}{13.37761pt}
\addtolength{\topmargin}{-1.37761pt}
% 数据经与Word模板比对得来

%--- 页眉页脚 ---%
\RequirePackage{fancyhdr}

\renewcommand{\headrulewidth}{0.725pt} % 页眉线宽
\renewcommand{\footrulewidth}{0pt} % 页脚线宽
\renewcommand{\chaptermark}[1]{\markboth{\thechapter\hspace{0.5em}#1}{}} % 修改章节标记,使\leftmark显示数字
\renewcommand{\sectionmark}[1]{\markright{\thesection\ #1}{}}

% 默认页面页眉页脚样式

\fancypagestyle{plain}{
    \fancyhf{}
    \fancyfoot[C]{\xiaowu\thepage}% 页码位于页脚居中

    % 左右两栏页眉,左为论文题目,右为当前章
    \fancyhead[L]{\wuhao{东北大学本科生毕业设计(论文)}}
    \fancyhead[R]{\wuhao{\leftmark}}
}
\pagestyle{plain}

% 页眉线
\makeatletter
\renewcommand{\headrule}{%
  % 让横线往下一点
  \vspace*{0.067em}
  % 也可以让页眉横线延长
  % 目前没延长,我觉得那不是官方模版的本意
  \hrule width 1.00\textwidth height \headrulewidth}
\makeatother

\fancypagestyle{statement}{
    \fancyhead[L]{\wuhao{东北大学本科生毕业设计(论文)}}
    \fancyhead[R]{\wuhao{郑重声明}}
    \fancyfoot[C]{}
}
\fancypagestyle{contents}{
    \fancyhead[L]{\wuhao{东北大学本科生毕业设计(论文)}}
    \setlength{\headsep}{22.5pt} % 增加页眉后高度
    \fancyhead[R]{\wuhao{目录}}
    \fancyfoot[C]{}
}
\fancypagestyle{cabstract}{
    \pagenumbering{Roman}
    \fancyhead[L]{\wuhao{东北大学本科生毕业设计(论文)}}
    \fancyhead[R]{\wuhao{摘要}}
}
\fancypagestyle{eabstract}{
    \fancyhead[L]{\wuhao{东北大学本科生毕业设计(论文)}}
    \fancyhead[R]{\wuhao{Abstract}}
}
\fancypagestyle{app2}{
    \fancyhead[L]{\wuhao{东北大学本科生毕业设计(论文)}}
    \fancyhead[R]{\wuhao{附录2:外文原文}}
    \fancyfoot[C]{}
}
\fancypagestyle{ack}{
    \fancyhead[L]{\wuhao{东北大学本科生毕业设计(论文)}}
    \fancyhead[R]{\wuhao{致谢}}
    \fancyfoot[c]{}
}

\fancypagestyle{listoffiguretable}{\fancyhead[R]{\kaiti\wuhao{插图与表格目录}}}

\RequirePackage{untilchapter}
% 首页右开的时候,全空的左页不留页眉页脚

%--- 段落格式 ---%
\RequirePackage{setspace}

%------- 封面 -------%
\newcommand{\coverit}[1]{{\rmfamily\textit{#1}}}
\newcommand{\CoverTitleText}{未设置标题}
\newcommand{\SetCoverTitle}[1]{\renewcommand{\CoverTitleText}{#1}}
\newcommand{\COVERtitle}{%
  \put(130.5,-329.65){%
    \parbox[t]{9.75cm}{%
      \zihao{-3}\setlength{\baselineskip}{31.3pt}\textbf{%
      \def\CJKglue{\hskip 0.1925em}% 逐字符插入间距,注意这个及以下这几个强调的注释不能删,而且%符号一定要紧贴右括号},否则会干预结果。。。玄学
      \def\\{\ignorespaces}% 让 \\ 不产生任何效果,这个注释也不能删
      \CoverTitleText}%
    }%
  }%
}
% 注:后来仔细读了读XeCJK的文档,应该是换行的时候识别成空格了
% 要么就全塞到一行里,要么就每次都加个%
% 换行时加%真是个好习惯呢~
\newcommand{\covertitle}{%
  \put(-17.03,-226.45){%
    \parbox[t]{15cm}{%
      \zihao{2}\heiti\centering
      \setlength{\baselineskip}{32pt}\CoverTitleText
    }%
  }%
}
\newcommand{\CoverIDText}{20227523}
\newcommand{\SetCoverID}[1]{\renewcommand{\CoverIDText}{#1}}
\newcommand{\coverid}{\put(58.6,3.9){\makebox(0,0)[c]{\heiti\wuhao\heitiarabic{\CoverIDText}}}}

\newcommand{\CoverCollegeText}{生命科学与健康学院}
\newcommand{\SetCoverCollege}[1]{\renewcommand{\CoverCollegeText}{#1}}
\newcommand{\COVERcollege}{%
  \put(130.5,-442.4){%
    \makebox(0,0)[l]{%
      \zihao{-3}\textbf{%
        \def\CJKglue{\hskip 0.1925em}% 这个注释不能删,会干预结果
        \CoverCollegeText
      }%
    }%
  }%
}
\newcommand{\covercollege}{\put(224.4,-389.35){\makebox(0,0)[l]{\zihao{-3}\CoverCollegeText}}}

\newcommand{\CoverMajorText}{生物工程}
\newcommand{\SetCoverMajor}[1]{\renewcommand{\CoverMajorText}{#1}}
\newcommand{\COVERmajor}{%
  \put(130.5,-478){%
    \makebox(0,0)[l]{%
      \zihao{-3}\textbf{%
        \def\CJKglue{\hskip 0.1925em}% 别删啊,说了别删
        \CoverMajorText
      }%
    }%
  }%
}
\newcommand{\covermajor}{\put(224.4,-420.65){\makebox(0,0)[l]{\zihao{-3}\CoverMajorText}}}

\newcommand{\CoverNameText}{薛定谔}
\newcommand{\SetCoverName}[1]{\renewcommand{\CoverNameText}{#1}}
\newcommand{\COVERname}{%
  \put(130.5,-513.6){%
    \makebox(0,0)[l]{%
      \zihao{-3}\textbf{%
        \def\CJKglue{\hskip 0.1925em}% 真别删
        \CoverNameText
      }%
    }%
  }%
}
\newcommand{\covername}{\put(224.4,-451.95){\makebox(0,0)[l]{\zihao{-3}\CoverNameText}}}

\newcommand{\CoverTeacherText}{郝仁}
\newcommand{\SetCoverTeacher}[1]{\renewcommand{\CoverTeacherText}{#1}}

\newcommand{\CoverTeacherTitleText}{教授}
\newcommand{\SetCoverTeacherTitle}[1]{\renewcommand{\CoverTeacherTitleText}{#1}}

\newcommand{\COVERteacher}{%
  \put(130.5,-549.2){%
    \makebox(0,0)[l]{%
      \zihao{-3}\textbf{%
        \def\CJKglue{\hskip 0.1925em}%
        \CoverTeacherText\hspace{1.385em}\CoverTeacherTitleText
      }%
    }%
  }%
}
\newcommand{\coverteacher}{\put(224.4,-483.25){\makebox(0,0)[l]{\zihao{-3}\CoverTeacherText\hspace{1em}\CoverTeacherTitleText}}}

\RequirePackage{zhnumber}
\newcommand{\CoverYearText}{2026}
\newcommand{\SetCoverYear}[1]{\renewcommand{\CoverYearText}{#1}}
\newcommand{\CoverMonthText}{6}
\newcommand{\SetCoverMonth}[1]{\renewcommand{\CoverMonthText}{#1}}

\newcommand{\COVERdate}{\put(222,-658.7){\makebox(0,0)[c]{\zihao{-3}\def\CJKglue{\hskip 0.07em}%
\zhdigits{\CoverYearText}年\hspace{0.6em}\zhnumber{\CoverMonthText}月}}}

\newcommand{\coverdate}{\put(197,-608.45){\makebox(0,0)[c]{\zihao{-3}\CoverYearText 年\CoverMonthText 月}}}

\newif\ifusebigcover
\usebigcoverfalse % 默认大封面关闭
\newcommand{\insertbigcover}{%
  \ifusebigcover
    \includepdf[
      pages=1,
      width=\paperwidth,
      height=\paperheight,
      pagecommand={
        \thispagestyle{empty}
        \begin{picture}(0,0)
         \put(29.5,-324.3){%
          \makebox(0,0)[l]{%
           \zihao{-3}\textbf{%
            \def\CJKglue{\hskip 0.1925em}
             \textbf{论文题目}
          }}}
          \COVERtitle
          \put(29.5,-442.4){%
          \makebox(0,0)[l]{%
           \zihao{-3}\textbf{%
            \def\CJKglue{\hskip 0.1925em}
             \textbf{学院名称}
          }}}
          \COVERcollege
          \put(29.5,-478){%
          \makebox(0,0)[l]{%
           \zihao{-3}\textbf{%
            \def\CJKglue{\hskip 0.1925em}
             \textbf{专业名称}
          }}}
          \COVERmajor
          \put(29.5,-513.6){%
          \makebox(0,0)[l]{%
           \zihao{-3}\textbf{%
            \def\CJKglue{\hskip 0.1925em}
             \textbf{学生姓名}
          }}}
          \COVERname
          \put(29.5,-549.2){%
          \makebox(0,0)[l]{%
           \zihao{-3}\textbf{%
            \def\CJKglue{\hskip 0.1925em}
             \textbf{指导教师}
          }}}
          \COVERteacher
          \COVERdate
        \end{picture}
      }
    ]{DoNotEdit/BIGCOVER.pdf}
    \AtEndDocument{%
    \huanye%
    \includepdf[%
    pages=-,%
    width=\paperwidth,%
    height=\paperheight]{DoNotEdit/BIGBACKCOVER.pdf}}
    \huanye
  \fi
}

% 已废弃
% \newcommand{\CoverDateText}{2026年6月} 
% \newcommand{\SetCoverDate}[1]{\renewcommand{\CoverDateText}{#1}}

%------- 郑重声明 -------%
% 签名路径与签名日期
\newcommand{\UserSignatureImage}{settings/signature.png}
\newcommand{\UserSignatureDate}{\today}

% 提供修改签名路径和日期的命令
\newcommand{\SetUserSignature}[2]{%
  \renewcommand{\UserSignatureDate}{#1}%
  \renewcommand{\UserSignatureImage}{#2}%
}

%------- 章节设置 -------%

% 修正chapter和(section/正文)之间的间距
% 原理我不懂,但调完能用就行
% 感谢@Izumi Sakai的方法
% https://ask.latexstudio.net/ask/question/17843.html
\RequirePackage{expl3} % LaTeX3
\makeatletter
\ExplSyntaxOn
\bool_new:N \l_@@_chapter_afterskip_bool
\hook_gput_code:nnn { cmd / chapter / before } {.}
  { \bool_gset_true:N \l_@@_chapter_afterskip_bool }
\hook_gput_code:nnn { cmd / @startsection / before } {.}
  { \bool_if:NT \l_@@_chapter_afterskip_bool { \vskip 0.25em } }
\DeclareDocumentCommand \banchapterskip {}
  { \bool_gset_false:N \l_@@_chapter_afterskip_bool }
\ExplSyntaxOff
\makeatother

% Heading格式
\makeatletter

\newif\ifheadingnumbertimes  % 控制编号字体
\newif\ifheadingtitletimes   % 控制标题内容字体

% 控制编号字体
\newcommand{\headingnumbertimes}[1]{%
  \def\@tempa{#1}%
  \def\@tempb{true}%
  \ifx\@tempa\@tempb
    \headingnumbertimestrue
  \else
    \headingnumbertimesfalse
  \fi
  \updateheadingformat
}

% 控制标题内容字体
\newcommand{\headingtitletimes}[1]{%
  \def\@tempa{#1}%
  \def\@tempb{true}%
  \ifx\@tempa\@tempb
    \headingtitletimestrue
  \else
    \headingtitletimesfalse
  \fi
  \updateheadingformat
}

\newcommand{\updateheadingformat}{%
  % 设置编号格式
  \ifheadingnumbertimes
    \ctexset{
      chapter/format            = \centering\zihao{-2}\heiti\timesrm,
      section/nameformat        = \zihao{4}\heiti\timesrm,
      subsection/nameformat     = \zihao{-4}\heiti\timesrm,
      subsubsection/nameformat  = \zihao{-4}\heiti\timesrm,
    }
  \else
    \ctexset{
      chapter/format            = \centering\zihao{-2}\heiti\heitiarabic,
      section/nameformat        = \zihao{4}\heiti\heitiarabic,
      subsection/nameformat     = \zihao{-4}\heiti\heitiarabic,
      subsubsection/nameformat  = \zihao{-4}\heiti\heitiarabic,
    }
  \fi
  % 设置标题内容格式
  \ifheadingtitletimes
    \ctexset{
      chapter/titleformat       = \zihao{-2}\heiti\timesrm,
      section/titleformat       = \zihao{4}\heiti\timesrm,
      subsection/titleformat    = \zihao{-4}\heiti\timesrm,
      subsubsection/titleformat = \zihao{-4}\heiti\timesrm,
    }
  \else
    \ctexset{
      chapter/titleformat       = \zihao{-2}\heiti\heitiarabic,
      section/titleformat       = \zihao{4}\heiti\heitiarabic,
      subsection/titleformat    = \zihao{-4}\heiti\heitiarabic,
      subsubsection/titleformat = \zihao{-4}\heiti\heitiarabic,
    }
  \fi
}
\makeatother

\pretocmd{\section}{\Needspace{4\baselineskip}}{}{}
\pretocmd{\subsection}{\Needspace{2\baselineskip}}{}{}
\pretocmd{\subsubsection}{\Needspace{2\baselineskip}}{}{}

% 设置默认值
\headingnumbertimes{true} % 编号用Times
\headingtitletimes{true}  % 标题内容用Times

% 其他ctex配置
\ctexset{
  chapter = {
    name        = {},
    number      = \arabic{chapter},
    aftername   = \hskip 0.5em,
    beforeskip  = -1.6em,
    afterskip   = 1.35em
  },
  section = {
    format      = \zihao{4}\heiti,
    aftername   = \hskip 0.5em,
    beforeskip  = 0.9em,
    afterskip   = 0.975em,
    aftertitle+ = \banchapterskip
  },
  subsection = {
    format      = \zihao{-4}\heiti,
    aftername   = \hskip 0.5em,
    beforeskip  = 0.525em,
    afterskip   = 0.4em
  },
  subsubsection = {
    format      = \zihao{-4}\heiti,
    aftername   = \hskip 0.5em,
    beforeskip  = 0.48em,
    afterskip   = -1.3em
  }
}

% 以下代码实现当某页以\subsection或\subsubsection起始时,修改页眉后距离
\RequirePackage{etoolbox}

% 定义subsection页面专用样式
\fancypagestyle{subsecpage}{%
  \fancyhf{}
  \fancyfoot[C]{\wuhao-\hspace{0.25em}\thepage\hspace{0.25em}-}%
  \fancyhead[L]{\wuhao 东北大学本科生毕业设计(论文)}%
  \fancyhead[R]{\wuhao \leftmark}%
  \setlength{\headsep}{12pt}%
  \setlength{\footskip}{25pt}%
}

% 定义subsubsection页面专用样式
\fancypagestyle{subsubsecpage}{%
  \fancyhf{}
  \fancyfoot[C]{\wuhao-\hspace{0.25em}\thepage\hspace{0.25em}-}%
  \fancyhead[L]{\wuhao 东北大学本科生毕业设计(论文)}%
  \fancyhead[R]{\wuhao \leftmark}%
  \setlength{\headsep}{14pt}%
  \setlength{\footskip}{29pt}%
}

% subsection
\makeatletter
% 定义标记变量,用于指示下一页是否需要应用特殊样式
\newif\if@nextpage@special@subsection
\@nextpage@special@subsectionfalse
% 在文档开始时重置标记
\AtBeginDocument{\global\@nextpage@special@subsectionfalse}
% 小节起始检测逻辑
\def\@subsec@begin{%
  \if@mainmatter
    %\typeout{DEBUG [SUBSEC]: pagetotal=\the\pagetotal, topskip=\the\topskip}%

    % 检测是否需要标记下一页应用特殊样式
    \ifdim\pagetotal>635pt % 为了防止过高的浮动体导致标记失败,设置了一个较宽容的值,可能会导致编译时间变长一些吧
     \ifdim\pagetotal<702pt
      \global\@nextpage@special@subsectiontrue
      %\typeout{*** [SUBSEC] Marking next page for special style ***}%
     \fi
    \fi

    % 检测当前页是否需要应用特殊样式
    \ifdim\pagetotal<0.1\topskip
      \thispagestyle{subsecpage}%
      %\typeout{*** [SUBSEC] Subsection at page top - style applied ***}%
    \else
      %\typeout{*** [SUBSEC] Subsection NOT at page top - style NOT applied ***}%
    \fi
  \fi
}
% 页面输出钩子,处理标记的页面
\AddToHook{shipout/background}{%
  \if@nextpage@special@subsection
    \thispagestyle{subsecpage}%
    %\typeout{*** [SUBSEC] Applying special style to marked page ***}%
    \global\@nextpage@special@subsectionfalse
  \fi
}
% 重定义命令,添加页面样式检测逻辑
\let\oldsubsection\subsection
\renewcommand{\subsection}{%
  \@ifstar
    {\@subsec@begin\oldsubsection*} % 处理带星号的小节
    {\@ifnextchar[{\@subsec@begin@opt}{\@subsec@begin@noopt}} % 处理可选参数
}
% 处理带可选参数的小节
\def\@subsec@begin@opt[#1]#2{%
  \@subsec@begin
  \oldsubsection[#1]{#2}%
}
% 处理不带可选参数的小节
\def\@subsec@begin@noopt#1{%
  \@subsec@begin
  \oldsubsection{#1}%
}
\makeatother

% 同理处理subsubsection
\makeatletter

% 标记变量
\newif\if@nextpage@special
\@nextpage@specialfalse

% 在文档开始时重置标记
\AtBeginDocument{\global\@nextpage@specialfalse}

% 起始检测逻辑
\def\@subsubsec@begin{%
  \if@mainmatter
    %\typeout{DEBUG: pagetotal=\the\pagetotal, topskip=\the\topskip}%
    
    % 检测是否需要标记下一页应用特殊样式
    \ifdim\pagetotal>620pt
     \ifdim\pagetotal<702pt
      \global\@nextpage@specialtrue
      %\typeout{*** Marking next page for special style ***}%
     \fi
    \fi
    
    % 检测当前页是否需要应用特殊样式
    \ifdim\pagetotal<0.1\topskip
      \thispagestyle{subsubsecpage}%
      %\typeout{*** Subsubsection at page top - style applied ***}%
    \else
      %\typeout{*** Subsubsection NOT at page top - style NOT applied ***}%
    \fi
  \fi
}

% 页面输出
\AddToHook{shipout/background}{%
  \if@nextpage@special
    \thispagestyle{subsubsecpage}%
    %\typeout{*** Applying special style to marked page ***}%
    \global\@nextpage@specialfalse
  \fi
}

% 命令重定义
\let\oldsubsubsection\subsubsection
\renewcommand{\subsubsection}{%
  \@ifstar
    {\@subsubsec@begin\oldsubsubsection*}% 处理带星号的子小节
    {\@ifnextchar[{\@subsubsec@begin@opt}{\@subsubsec@begin@noopt}}% 处理可选参数
}

% 带可选参数的
\def\@subsubsec@begin@opt[#1]#2{%
  \@subsubsec@begin
  \oldsubsubsection[#1]{#2}%
}

% 不带可选参数
\def\@subsubsec@begin@noopt#1{%
  \@subsubsec@begin
  \oldsubsubsection{#1}%
}

\makeatother

% 这段实现在section后判断:正文-->添加vskip;subsection-->无操作
% 感谢Claude的帮助
% 感谢@Sagittarius Rover的灵感
% https://ask.latexstudio.net/ask/question/17843.html
% 这问题卡了我三天......
\makeatletter
% 定义标志来跟踪当前是否为section、subsection或subsubsection
\newif\if@insection
\newif\if@insubsection
\newif\if@insubsubsection
\@insectionfalse
\@insubsectionfalse
\@insubsubsectionfalse
% 保存原始的@startsection命令
\let\orig@startsection\@startsection
% 重新定义@startsection
\def\@startsection#1#2#3#4#5#6{%
  % 重置所有标志
  \@insectionfalse
  \@insubsectionfalse
  \@insubsubsectionfalse
  % 检查级别并设置相应标志
  \ifnum#2=1\relax
    \@insectiontrue
  \else\ifnum#2=2\relax
    \@insubsectiontrue
  \else\ifnum#2=3\relax
    \@insubsubsectiontrue
  \fi\fi\fi
  % 调用原始@startsection
  \orig@startsection{#1}{#2}{#3}{#4}{#5}{#6}%
}
% 保存原始的@afterheading
\let\orig@afterheading\@afterheading
% 重新定义@afterheading
\def\@afterheading{%
  \@nobreaktrue
  \everypar{%
    \if@nobreak
      \@nobreakfalse
      \clubpenalty \@M
      \if@afterindent \else
        {\setbox\z@\lastbox}%
      \fi
      % 检查section:如果是section且下一个不是subsection,则插入vskip
      \if@insection
        \@ifnextchar\subsection{}{%
          \vskip -1.73em}%
        \@insectionfalse
      % 检查subsection:如果是subsection且下一个不是subsubsection,则插入vskip
      \else\if@insubsection
        \@ifnextchar\subsubsection{}{%
          \vskip -1.75em}% 9.30 原1.75
        \@insubsectionfalse
      % 检查subsubsection:如果是subsubsection且下一个不是paragraph,则插入vskip
      \else\if@insubsubsection
        \@ifnextchar\paragraph{}{%
          \vskip 0em}%
        \@insubsubsectionfalse
      \fi\fi\fi
    \else
      \clubpenalty \@clubpenalty
    \fi
    \everypar{}%
  }%
}

\let\orig@begin\begin
% 给subsubsection后的紧跟列表添加额外间距
\renewcommand{\begin}[1]{%
  % 检查是否是列表环境且刚刚处理了subsubsection
  \if@insubsubsection
    \def\@tempa{#1}%
    \def\@tempb{itemize}%
    \def\@tempc{enumerate}%
    \ifx\@tempa\@tempb
      \vskip 1.88em  % itemize前的额外间距
      \@insubsubsectionfalse
    \else\ifx\@tempa\@tempc
      \vskip 1.88em  % enumerate前的额外间距
      \@insubsubsectionfalse
    \fi\fi
  \fi
  % 调用原始\begin命令
  \orig@begin{#1}%
}

\makeatother

%------- 中文摘要 -------%
\newenvironment{cabstract}{
    \setcounter{secnumdepth}{-1}
    \chapter*{{
    \xiaoer
    \vspace*{-23.575pt}\\
    \hspace*{0.2em}摘\hspace{1em}要}
    \vspace*{-4.5pt}}
    \addcontentsline{toc}{chapter}{摘要}
    \songti\xiaosi
    \setlength{\baselineskip}{23.086043pt}
    \thispagestyle{cabstract}
}{
    \setcounter{secnumdepth}{3}
}

%--- 中文关键词 ---%
\def\ckeywords#1{
    \vskip 1.95em
    \begin{spacing}{1}
      \noindent\heiti\xiaosi\hspace*{-0.2em}\textbf{关键词:}\songti\xiaosi #1
    \end{spacing}
}

%------- 英文摘要 -------%
\newenvironment{eabstract}{
    \setcounter{secnumdepth}{-1}
    \chapter*{{
    \xiaoer
    \vspace*{-23.575pt}\\
    \textrm{\hspace{0.2em}\textbf{ABSTRACT}}}
    \vspace*{-4.5pt}}
    \addcontentsline{toc}{chapter}{ABSTRACT}
    \xiaosi
    \setlength{\baselineskip}{23.086043pt}
    \thispagestyle{eabstract}
}{
    \setcounter{secnumdepth}{3}
}

%--- 英文关键词 ---%
\def\ekeywords#1{
    \vskip 1.65em
    \begin{spacing}{1}
      \noindent\xiaosi\textbf{Key words:~}\xiaosi #1
    \end{spacing}
    \pagestyle{eabstract}
}

%------- 目录 -------%
\setcounter{secnumdepth}{3}
\setcounter{tocdepth}{3}

\ctexset{contentsname={\hspace*{-0.2em}目\hspace{1em}录}}

\RequirePackage{tocloft}

\renewcommand{\cfttoctitlefont}{\hfill\xiaoer\heiti}
\renewcommand{\cftaftertoctitle}{\hfill}

\setlength{\cftbeforetoctitleskip}{-9pt} % 目录标题与页面顶部间距
\setlength{\cftaftertoctitleskip}{-1.75pt} % 目录标题与内容间距

\renewcommand{\cftdot}{.}
%\renewcommand{\cftdot}{\CJKTimes…} % 官方示例的省略号引导线,但效果不好,启用的话替换掉上一行即可
\renewcommand{\cftdotsep}{0.2} % 点间距
\renewcommand{\cftchapdotsep}{\cftdotsep}

% Thanks to @Sagittarius Rover、@雾月
% https://ask.latexstudio.net/ask/question/8102.html
\makeatletter%%
\renewcommand{\cftchapfillnum}[1]{%
  {\cftchapleader}\nobreak
  \makebox{\cftchappagefont #1}\cftchapafterpnum\par
}
\renewcommand{\cftsecfillnum}[1]{%
  {\cftsecleader}\nobreak
  \makebox{\cftsecpagefont #1}\cftsecafterpnum\par
}
\renewcommand{\cftsubsecfillnum}[1]{%
  {\cftsubsecleader}\nobreak
  \makebox{\cftsubsecpagefont #1}\cftsubsecafterpnum\par
}
\renewcommand{\cftsubsubsecfillnum}[1]{%
  {\cftsubsubsecleader}\nobreak
  \makebox{\cftsubsubsecpagefont #1}\cftsubsubsecafterpnum\par
}
\renewcommand{\cftparafillnum}[1]{%
  {\cftparaleader}\nobreak
  \makebox{\cftparapagefont #1}\cftparaafterpnum\par
}
\makeatother

%--- 目录项间距设置 --- %

% 设置常规间距
\setlength{\cftbeforechapskip}{21pt}
\setlength{\cftbeforesecskip}{13pt}
\setlength{\cftbeforesubsecskip}{2.5ex}
\setlength{\cftbeforesubsubsecskip}{2.5ex}

% 以下代码使目录中每级节与其后的第一个子节之间额外增加间距
\makeatletter
% 定义所有标记
\newif\if@afterchapter
\@afterchapterfalse
\newif\if@aftersection
\@aftersectionfalse
\newif\if@aftersubsection
\@aftersubsectionfalse

% chapter渲染逻辑
\let\old@l@chapter\l@chapter
\renewcommand{\l@chapter}[2]{%
  \old@l@chapter{#1}{#2}%
  \global\@afterchaptertrue
  \global\@aftersectionfalse % 清除其他标记
  \global\@aftersubsectionfalse
}

% section渲染逻辑
\let\old@l@section\l@section
\renewcommand{\l@section}[2]{%
  \if@afterchapter
    \addvspace{2.75pt} % chapter后的额外间距
    \global\@afterchapterfalse
  \fi
  \old@l@section{#1}{#2}%
  \global\@aftersectiontrue % 为subsection设置标记
  \global\@aftersubsectionfalse % 清除subsection标记
}

% subsection渲染逻辑
\let\old@l@subsection\l@subsection
\renewcommand{\l@subsection}[2]{%
  \if@aftersection
    \addvspace{-0.65pt} % section后的额外间距
    \global\@aftersectionfalse
  \fi
  \old@l@subsection{#1}{#2}%
  \global\@aftersubsectiontrue % 为subsubsection设置标记
}

% subsubsection渲染逻辑
\let\old@l@subsubsection\l@subsubsection
\renewcommand{\l@subsubsection}[2]{%
  \if@aftersubsection
    \addvspace{0pt} % subsection后的额外间距
    \global\@aftersubsectionfalse
  \fi
  \old@l@subsubsection{#1}{#2}%
}
\makeatother

% 以下代码实现低级别标题后接chapter插入额外间距
\makeatletter
% 定义标记,用于记录最后一个渲染的标题级别
\newif\if@lastwassection
\@lastwassectionfalse
\newif\if@lastwassubsection
\@lastwassubsectionfalse
\newif\if@lastwassubsubsection
\@lastwassubsubsectionfalse

% 重新定义chapter,检测之前的标题级别
\let\oldold@l@chapter\l@chapter
\renewcommand{\l@chapter}[2]{%
  % 检测并添加相应间距
  \if@lastwassubsubsection
    \addvspace{15pt} % subsubsection-chapter
    \global\@lastwassubsubsectionfalse
  \else
    \if@lastwassubsection
      \addvspace{4pt} % subsection-chapter
      \global\@lastwassubsectionfalse
    \else
      \if@lastwassection
        \addvspace{-4.25pt} % section-chapter
        \global\@lastwassectionfalse
      \fi
    \fi
  \fi
  % 调用原始的chapter渲染
  \oldold@l@chapter{#1}{#2}%
  % 清除所有标记
  \global\@lastwassectionfalse
  \global\@lastwassubsectionfalse
  \global\@lastwassubsubsectionfalse
}

% 更新section,设置最后标题标记
\let\oldold@l@section\l@section
\renewcommand{\l@section}[2]{%
  % 清除所有标记
  \global\@lastwassectionfalse
  \global\@lastwassubsectionfalse
  \global\@lastwassubsubsectionfalse
  % 调用原始的section渲染
  \oldold@l@section{#1}{#2}%
  % 设置当前级别标记
  \global\@lastwassectiontrue
}

% 更新subsection,设置最后标题标记
\let\oldold@l@subsection\l@subsection
\renewcommand{\l@subsection}[2]{%
  % 清除所有标记
  \global\@lastwassectionfalse
  \global\@lastwassubsectionfalse
  \global\@lastwassubsubsectionfalse
  % 调用原始的subsection渲染
  \oldold@l@subsection{#1}{#2}%
  % 设置当前级别标记
  \global\@lastwassubsectiontrue
}

% 更新subsubsection,设置最后标题标记
\let\oldold@l@subsubsection\l@subsubsection
\renewcommand{\l@subsubsection}[2]{%
  % 清除所有标记
  \global\@lastwassectionfalse
  \global\@lastwassubsectionfalse
  \global\@lastwassubsubsectionfalse
  % 调用原始的subsubsection渲染
  \oldold@l@subsubsection{#1}{#2}%
  % 设置当前级别标记
  \global\@lastwassubsubsectiontrue
}
\makeatother

%--- 目录项字体格式 ---%
\makeatletter
\def\@true{true}
% Chapter编号字体
\newcommand{\tocchapternumbertimes}[1]{%
  \def\@tocchapternumbertimes{#1}%
}
\def\@tocchapternumbertimes{true} % 默认值
% Chapter标题字体
\newcommand{\tocchaptertitletimes}[1]{%
  \def\@tocchaptertitletimes{#1}%
}
\def\@tocchaptertitletimes{true} % 默认值
% Chapter页码字体
\newcommand{\tocchapterpagetimes}[1]{%
  \def\@tocchapterpagetimes{#1}%
}
\def\@tocchapterpagetimes{true} % 默认值
\AtBeginDocument{%
  % Chapter编号字体
  \ifx\@tocchapternumbertimes\@true
    \renewcommand{\cftchappresnum}{\timesrm}%
  \else
    \renewcommand{\cftchappresnum}{\heitiarabic}%
  \fi
  %
  % Chapter标题字体
  \ifx\@tocchaptertitletimes\@true
    \renewcommand{\cftchapfont}{\sihao\heiti\timesrm}%
  \else
    \renewcommand{\cftchapfont}{\sihao\heiti\SimHeiWestern}%
  \fi
  %
  % Chapter页码字体
  \ifx\@tocchapterpagetimes\@true
    \renewcommand{\cftchappagefont}{\timesrm}%
  \else
    \renewcommand{\cftchappagefont}{\heiti\SimHeiWestern}%
  \fi
}
\makeatother

\renewcommand{\cftsecfont}{\xiaosi\songti}
\renewcommand{\cftsubsecfont}{\xiaosi\songti}
\renewcommand{\cftsubsubsecfont}{\xiaosi\songti}


%--- 目录项缩进 ---%

% 设置各级标题与边距的缩进
\setlength{\cftchapindent}{-0.25em}
\setlength{\cftsecindent}{0em}
\setlength{\cftsubsecindent}{0em}
\setlength{\cftsubsubsecindent}{0em}

\setlength{\cftchapnumwidth}{1.75em} % chapter数字宽度
\renewcommand{\cftchapaftersnum}{} % chapter后不带点

% 两个数字参数分别是各级的缩进距离和数字宽度
\cftsetindents{section}{1.75em}{2.25em}
\cftsetindents{subsection}{4.05em}{3em}
\cftsetindents{subsubsection}{7.1em}{3.75em}
\renewcommand{\cftchapleader}{\cftdotfill{\cftchapdotsep}}

% 页眉页脚
\newcommand{\contents}{
\tableofcontents
\thispagestyle{contents}}

%------- 列表 -------%
\RequirePackage{enumitem}
%--- 无序列表 ---%
\setlist[itemize]{%
  itemsep=0pt, % 每一项之间的垂直间距
  topsep=0pt, % 列表上下方与正文的垂直间距
  parsep=0pt, % 段落间距
  partopsep=0pt, % 段落顶部附加间距
  left=2em % 缩进距离(控制整体左移或右移)
}

%--- 有序列表 ---%
\RequirePackage{multicol}

\setlist[enumerate]{%
  itemsep=0pt,
  topsep=0pt,
  parsep=0pt,
  partopsep=0pt,
  left=2em
}

\SetEnumitemKey{shuzi}{
    label=\arabic*.,
    listparindent=2em,
    labelsep=0.25em,
    itemindent=3em
}

\SetEnumitemKey{kuohao}{
    label=(\arabic*),
    listparindent=2em,
    labelsep=0pt,
    itemindent=4em
}

\RequirePackage{circledsteps} % \Circled
\newcommand{\Quan}[1]{%
  \Circled{%
    \scalebox{0.45}[1]{%
      \songtiarabic{#1}%
    }%
  }%
}

\newcommand{\quansplit}[1]{%
    \ifnum#1<10
        \quan{#1}%
    \else\ifnum#1<100
        \edef\tempnum{\number#1}%
        \expandafter\quansplithelper\tempnum\relax
    \else
        \Quan{#1}%
    \fi\fi
}

\def\quansplithelper#1#2\relax{%
    \ifx\relax#2\relax
        \quan{#1}%
    \else
        \quan[#1]{#2}%
    \fi
}

\SetEnumitemKey{quan}{
    label=\hspace{1em}\protect\quansplit{\arabic*},
    listparindent=2em,
    labelwidth=2em,
    labelsep=0pt,
    itemindent=3em
}

\RequirePackage{alphalph}

% 处理enumitem星号机制
\makeatletter
\newcommand{\AlphAlphEnum}[1]{\AlphAlph{\value{#1}}}
\newcommand{\alphalphEnum}[1]{\alphalph{\value{#1}}}

% 注册为enumitem的计数器
\AddEnumerateCounter{\AlphAlphEnum}{\AlphAlph}{AA}
\AddEnumerateCounter{\alphalphEnum}{\alphalph}{aa}
\makeatother

% 定义键
\SetEnumitemKey{daxie}{
    label=\AlphAlphEnum*.,
    listparindent=2em,
    labelsep=0.25em,
    itemindent=3.25em
}
\SetEnumitemKey{xiaoxie}{
    label=\alphalphEnum*.,
    listparindent=2em,
    labelsep=0.25em,
    itemindent=3.25em
}

%------- 图表 -------%

%--- 题注格式 ---%
\renewcommand{\captionfont}{\fontsize{10.5pt}{10.5pt}\songti\selectfont}
\DeclareCaptionLabelSeparator{oneemsep}{\hspace{1em}}
\captionsetup[table]{
  font={small},
  labelsep=oneemsep, % 去掉冒号,改为1em空格
  skip=0pt,
  belowskip = 0pt
}
\captionsetup[figure]{
    labelsep=oneemsep,
    skip=17.9pt,
    belowskip=-9.9pt %-15.74
}

\NewDocumentCommand{\xucaption}{m}{%
  \caption*{图\ref{#1}(续)}} % 续图标题

% 注
\makeatletter

\newif\if@afterzhu
\@afterzhufalse

\newcommand{\zhu}[1]{%
  \vspace*{2.5pt}%
  {\wuhao\noindent #1\par\vskip -0.13075em}
  \global\@afterzhutrue%
  % 设置下一段开始时清除标志
  \everypar{\global\@afterzhufalse\everypar{}}%
}

\let\zhu@orig@section\section
\let\zhu@orig@subsection\subsection  
\let\zhu@orig@subsubsection\subsubsection

\renewcommand{\section}{%
  \if@afterzhu%
    \vspace{0.105em}%
    \global\@afterzhufalse%
    \everypar{} % 清除everypar设置
  \fi%
  \zhu@orig@section%
}

\renewcommand{\subsection}{%
  \if@afterzhu%
    \vspace{0.043em}%
    \global\@afterzhufalse%
    \everypar{}%
  \fi%
  \zhu@orig@subsection%
}

\renewcommand{\subsubsection}{%
  \if@afterzhu%
    \vspace{0.043em}%
    \global\@afterzhufalse%
    \everypar{}%
  \fi%
  \zhu@orig@subsubsection%
}
\makeatother

% 表注
\newcommand{\biaozhu}{\zhu}

% 图注
\newcommand{\tuzhu}{\zhu}

\makeatletter
\let\orig@section\section
\let\orig@subsection\subsection
\let\orig@subsubsection\subsubsection
\newif\if@notuzhu
\@notuzhufalse

\renewcommand{\tuzhu}{%
  \@ifnextchar\bgroup{\@tuzhu@witharg}{\@tuzhu@noarg}%
}

% 当后有参数时
\def\@tuzhu@witharg#1{%
  \def\@tuzhu@arg{#1}%
  \ifx\@tuzhu@arg\@empty
    % 参数为空,执行notuzhu
    \@tuzhu@notuzhu
  \else
    % 参数非空,调用原命令
    \zhu{#1}%
  \fi
}

% 当后无参数
\def\@tuzhu@noarg{%
  \@tuzhu@notuzhu
}

% notuzhu检查后续命令
\def\@tuzhu@notuzhu{%
  \futurelet\@notuzhu@next\@notuzhu@check
}

% 检查下一个是否是 section/subsection/subsubsection
\def\@notuzhu@check{%
  \ifx\@notuzhu@next\section
    \@notuzhutrue
  \else\ifx\@notuzhu@next\subsection
    \@notuzhutrue
  \else\ifx\@notuzhu@next\subsubsection
    \@notuzhutrue
  \else
    % 如果不是,立即重置标志
    \@notuzhufalse
  \fi\fi\fi
}

\renewcommand{\section}{%
  \if@notuzhu
    \vspace{-0.625em}
    \@notuzhufalse % 使用后立即重置标志
  \fi
  \orig@section
}

\renewcommand{\subsection}{%
  \if@notuzhu
    \vspace{-0.7em}
    \@notuzhufalse
  \fi
  \orig@subsection
}

\renewcommand{\subsubsection}{%
  \if@notuzhu
    \vspace{-0.665em}
    \@notuzhufalse
  \fi
  \orig@subsubsection
}

\makeatother

% 以下代码调整续图题后接正文时的间距

\newif\ifusedxucaption
\usedxucaptionfalse

\let\oldxucaption\xucaption
\renewcommand{\xucaption}[1]{%
    \global\usedxucaptiontrue % 设置标志为true
    \oldxucaption{#1}%
}

\let\oldcaption\caption
\renewcommand{\caption}[1]{%
    \global\usedxucaptionfalse % 设置标志为false
    \oldcaption{#1}%
}

% 继续修改\@tuzhu@witharg,在调用\zhu后添加检测
% 俺咋一直打补丁呢
\makeatletter
\def\@tuzhu@witharg#1{%
  \def\@tuzhu@arg{#1}%
  \ifx\@tuzhu@arg\@empty
    \@tuzhu@notuzhu
  \else
    \zhu{#1}%
    \ifusedxucaption%
        \vspace*{0.36em} % 如果使用了\xucaption,添加间距
    \fi%
    \global\usedxucaptionfalse % 重置标志
  \fi
}
\makeatother

%--- 表格 ---%
\RequirePackage{array}
\RequirePackage{tabularx}
\RequirePackage{
   longtable,
   xltabular, % 长表格
   booktabs,  % 三线表
   makecell,
   multirow,  % 跨行表格
   diagbox    % 斜线表头
}
% 定义表格(非续表)页面专用样式
\fancypagestyle{bgpage}{%
  \fancyhf{}
  \fancyfoot[C]{\wuhao-\hspace{0.25em}\thepage\hspace{0.25em}-}%
  \fancyhead[L]{\wuhao 东北大学本科生毕业设计(论文)}%
  \fancyhead[R]{\wuhao\leftmark}%
  \setlength{\headsep}{12pt}%
  \setlength{\footskip}{34.5pt}%
}
% 定义续表页面专用样式
\fancypagestyle{xbpage}{%
  \fancyhf{}
  \fancyfoot[C]{\wuhao-\hspace{0.25em}\thepage\hspace{0.25em}-}%
  \fancyhead[L]{\wuhao 东北大学本科生毕业设计(论文)}%
  \fancyhead[R]{\wuhao\leftmark}%
  \setlength{\headsep}{16pt}%
  \setlength{\footskip}{30.5pt}%
}

\newcommand{\tabsize}{\wuhao}

% 列类型Y:自动列宽,居中
\newcolumntype{Y}{>{\centering\arraybackslash}X}
\renewcommand{\tabularxcolumn}[1]{m{#1}} % 修改X为垂直居中

\newcolumntype{Z}{>{\hfill\arraybackslash}X}

\makeatletter

\newif\if@biaoge@firstpage

\newcommand{\biaoge}[7][1.96]{%  % 添加可选参数,默认值1.96
  \tabsize
  \setlength{\aboverulesep}{0pt}
  \setlength{\belowrulesep}{0pt}
  \renewcommand{\arraystretch}{#1}%  % 使用可选参数控制行距
  \setlength{\extrarowheight}{-1.25pt} % 额外行高
  \vspace{-\abovecaptionskip+0.167pt}
  \def\kycaption{#2}%  % 原#1改为#2
  \def\kylabel{#3}%    % 原#2改为#3...
  \let\old@LT@output\LT@output
  % 每次调用重置标记
  \global\@biaoge@firstpagetrue
  \def\LT@output{%
    \if@biaoge@firstpage
      \global\@biaoge@firstpagefalse % 首页输出后,标记为false
    \else
      \thispagestyle{xbpage}
    \fi
    \old@LT@output
  }%
  % 检测当前页面位置,决定是否应用特殊样式
  \@table@begin
  \begin{xltabular}{\textwidth}{*{#4}{#5}}
    \caption[\kycaption]{\kycaption\label{\kylabel}\vspace*{-18pt}} \\
    \toprule
    #6 \\
    \midrule
    \endfirsthead
    % 续表部分调整间距
    \multicolumn{#4}{c}{%  % 原#3改为#4
      续表~\ref{#3}~~#2\vspace*{-3.5pt} % 原#2改为#3,原#1改为#2
    } \\
    \toprule
    #6 \\
    \midrule
    \endhead
    \bottomrule
    \endfoot
    \bottomrule
    \endlastfoot
    #7
  \end{xltabular}
  \let\LT@output\old@LT@output
  \vspace*{-17pt} % 表格后间距
  \xiaosi
  \setlength{\baselineskip}{23.086043pt}
}

%--- 手动列宽表格 ---% 
\newcolumntype{n}[1]{>{\centering\arraybackslash}m{#1}} % 固定列宽且居中
\makeatletter

\newif\if@sdbiaoge@firstpage

\makeatletter
\newcommand{\sdbiaoge}[6][1.96]{%
  \tabsize
  \setlength{\aboverulesep}{0pt}
  \setlength{\belowrulesep}{0pt}
  \renewcommand{\arraystretch}{#1}%
  \setlength{\extrarowheight}{-1.25pt}
  \vspace{-\abovecaptionskip+12pt}%
  \def\whcaption{#2}%
  \def\whlabel{#3}%
  \let\old@LT@output\LT@output
  \global\@sdbiaoge@firstpagetrue
  \def\LT@output{%
    \if@sdbiaoge@firstpage
      \global\@sdbiaoge@firstpagefalse
    \else
      \thispagestyle{xbpage}
    \fi
    \old@LT@output
  }%
  \@table@begin
  \begingroup
  \setlength{\LTleft}{0pt}%
  \setlength{\LTright}{0pt}%
  \setlength{\LTpre}{0pt}%
  \setlength{\LTpost}{0pt}%
  \begin{longtable}{@{\extracolsep{\fill}}#4@{}}
    \caption[\whcaption]{\whcaption\label{\whlabel}\vspace*{-18pt}} \\
    \noalign{\vspace{0pt}}
    \toprule
    #5 \\
    \midrule
    \endfirsthead
    \wuhao\makebox[\textwidth][c]{续表~\ref{\whlabel}\hspace{1em}\whcaption\vspace*{0pt}} \\
    \noalign{\vspace{-4pt}}
    \toprule
    #5 \\
    \midrule
    \endhead
    \bottomrule
    \endfoot
    #6
  \end{longtable}
  \endgroup
  \let\LT@output\old@LT@output
  \vspace*{-5pt}%
  \xiaosi
  \setlength{\baselineskip}{23.086043pt}
}
\makeatother

% 当表格出现在页面顶部时,应用表格页面样式
\makeatletter
\newif\if@nextpage@special@table
\@nextpage@special@tablefalse
\AtBeginDocument{\global\@nextpage@special@tablefalse}
\def\@table@begin{%
  %\typeout{DEBUG [TABLE]: pagetotal=\the\pagetotal, topskip=\the\topskip}%
\begingroup
  \dimen0=\pagetotal
  \ifdim\dimen0>620pt
    \ifdim\dimen0<702pt
      \global\@nextpage@special@tabletrue
      %\typeout{*** [TABLE] Marking next page for special style ***}%
    \fi
  \fi
\endgroup

  \ifdim\pagetotal<0.1\topskip
    \thispagestyle{bgpage}%
    % \typeout{*** [TABLE] \thepage Table at page top - style applied: \the\pagetotal<0.5\the\topskip***}%
  \else
    % \typeout{*** [TABLE] \thepage Table NOT at page top - style NOT applied ***}%
  \fi
}
\AddToHook{shipout/background}{%
  \if@nextpage@special@table
    \thispagestyle{bgpage}%
    % \typeout{*** [TABLE] Applying special style to marked page ***}%
    \global\@nextpage@special@tablefalse
  \fi
}
\let\oldbiaoge\biaoge
\let\oldsdbiaoge\sdbiaoge
\renewcommand{\biaoge}{%
  \@table@begin
  \oldbiaoge
}
\renewcommand{\sdbiaoge}{%
  \@table@begin
  \oldsdbiaoge
}
\makeatother

%------- 尾注 -------%
% 尾注每页重新计数用,https://cloud.tencent.com/developer/ask/sof/109005693
\RequirePackage{perpage}
\renewcommand{\footnoterule}{
  \vspace{1em} % 分割线前的垂直间距
  \hrule width 0.35\textwidth height 1pt % 设置分割线宽度和粗细
  \vspace{0.6em} % 分割线后的垂直间距
}

\xeCJKsetcharclass{`①}{`⑩}{1} % ①~⑩ Unicode字符支持

\newcommand{\quangeweishu}[1]{% 这个没用
    \kern0.0037em%
    \tikz[baseline=(char.base)]{
        \node[shape=circle, draw, inner sep=0.35pt, minimum size=0.965em, line width=0.03em] (char) {\songtiarabic{#1}};
    }%
    \kern0.0037em%
}

\newcommand{\quanshi}[2]{% For 10-19
    \kern0.04em%
    \tikz[baseline=(char.base)]{
        \node[shape=circle, draw, inner sep=0, minimum size=0.9em, line width=0.0333em] (char) {%
            \scalebox{0.95}[1]{\fontsize{0.82em}{0.82em}\selectfont\songtiarabic{#1\kern-0.12em#2}}};
    }%
    \kern0.04em%
}

\newcommand{\quanershi}[2]{% For 20-99
    \kern0.04em%
    \tikz[baseline=(char.base)]{
        \node[shape=circle, draw, inner sep=-0.75pt, minimum size=0.9em, line width=0.0333em] (char) {%
            \scalebox{0.85}[1]{\fontsize{0.82em}{0.82em}\songtiarabic{#1\kern-0.05em#2}}};
    }%
    \kern0.04em%
}

\newcommand{\quan}[2][0]{%
    \ifnum#1=0\relax% 只有一个参数(#2)
        \ifcase#2\relax
        \or \char"2460 % 1
        \or \char"2461 % 2
        \or \char"2462 % 3
        \or \char"2463 % 4
        \or \char"2464 % 5
        \or \char"2465 % 6
        \or \char"2466 % 7
        \or \char"2467 % 8
        \or \char"2468 % 9
        \fi
    \else% 有两个参数
        \ifnum#1=1\relax
          \ifnum#2>0\relax
            \quanshi{#1}{#2} % 11-19
        \else
            \char"2469 % 10
          \fi
        \else
          \quanershi{#1}{#2} % 20-99
        \fi
    \fi
}

% 脚注使用带圈数字
\newcommand*\quanctr[1]{%
  \ifnum\value{#1}<10\relax
    % 1-9 \quan{个位数}
    \protect\quan{\number\value{#1}}%
  \else\ifnum\value{#1}<20\relax
    % 10-19 \quan[1]{个位数}
    \protect\quan[1]{\number\numexpr\value{#1}-10\relax}%
  \else\ifnum\value{#1}<100\relax
    % 20-99 \quan[十位数]{个位数}
    \protect\quan[\number\numexpr\value{#1}/10\relax]{\number\numexpr\value{#1}-(\value{#1}/10)*10\relax}%
  \else
    % 100以上
    \number\Quan{\value{#1}}%
  \fi\fi\fi
}

\renewcommand*\thefootnote{\quanctr{footnote}}

%------- 代码 -------%
\RequirePackage{fancyvrb} % 用于verbatim的高级设置
\RequirePackage{fvextra} % 支持breaklines选项
\fvset{breaklines=true}
\RequirePackage{framed} % 用于添加边框

\RecustomVerbatimEnvironment{verbatim}{Verbatim}{frame = single, numbers=left, tabsize = 4, fontsize=\footnotesize, showspaces=false} % 调整字间距,加框

\let\oldverbatim\verbatim
\let\endoldverbatim\endverbatim

\RenewDocumentEnvironment{verbatim}{}{
  \oldverbatim
}{
  \endoldverbatim
  \vspace{-1em}
}

%------- 参考文献 -------%
\RequirePackage[
    backend = biber,
    % 尽管Word模板里说的是2005,但实际采用的是2015标准
    % 为了使中西文文献使用不同标点,采用最新的2025版宏包
    style = gb7714-2025,
    % 按出现顺序编号
    sorting = none,
    % uppercase则人名全大写;lowercase则仅首字母大写
    gbnamefmt = lowercase,
    % 当true时,会国标的要求,使用"佚名""Anon""出版地不详"等补充信息
    % 学校要求不添加这些信息
    gbnoauthor = false,
    % 使用连续出版物析出文献起止范围的卷期
    gbpub = true,
    % 中文文献用全角,英文文献用半角,是2025版本的新功能,目前存在bug(已自行修复)
    gbpunctwidth = bylan,
    % url和DOI号建议保留,因为网络文献需要列出,建议在bib文件中手动删去url和DOI号
    % 显示链接
    url = true,
    % 显示DOI号
    doi = true
]{biblatex}%
\addbibresource{data/references.bib} % bib文件名
\AtBeginBibliography{%
\setlength{\baselineskip}{23.086043pt}%
}
\setlength{\bibitemsep}{0pt}
\setlength{\biblabelsep}{7.4mm}
% \DeclareFieldFormat[article]{journaltitle}{\mkbibemph{#1}} % 期刊名斜体,可选启用
% \DeclareFieldFormat[article]{volume}{\mkbibemph{#1}} % 卷数斜体,可选启用
% \DeclareFieldFormat{date}{\textbf{#1}} % 年份斜体,可选启用
\DeclareCiteCommand{\citenum} % 自订“裸数字”命令
  {\usebibmacro{prenote}}
  {\printtext[bibhyperref]{\printfield{labelnumber}}} % 仅输出序号并带链接
  {\multicitedelim}
  {\usebibmacro{postnote}}
\setcounter{gbrefcompress}{3} % 连续3篇文献才开始压缩
\def\gbpunctcommacite{\addcomma} % 修改宏包,使引用时用半角逗号
% 修改全角括号为半角括号
\def\gbpunctprl{(}%
\def\gbpunctprr{)}%
\def\CharParenLeft{\iffieldequalstr{userd}{chinese}{(}{(}}
\def\CharParenRight{\iffieldequalstr{userd}{chinese}{)}{)}}%
% 以下五行代码解决了gbpunctwidth=bylan的bug,文献标题后的标点问题
% 参考Hu Zhenzhen,20180405,为texlive2017以上版本中的beamer兼容性做的处理
% \DeclareFieldFormat{title}{#1}%
% \DeclareFieldFormat[article,patent,thesis,unpublished]{title}{#1}
% \makeatletter
% \AtBeginDocument{\patchcmd{\abx@macro@title}{\newunitpunct}{}{}{}}{}
% \makeatother
% 下面这行代码解决了连续出版物标识符[J]后多一个空格的bug
\def\gbpunctdot{\unskip\mbox{.}\allowbreak}
% 以上问题及解决办法已反馈至GibHub官方仓库,注释掉则代表已经被官方采纳

% 设置是否允许条目跨页
\newif\ifbibbreakentry
\bibbreakentryfalse % 默认 false,即不跨页

\newcommand{\bibbreakentry}[1]{%
  \def\temparg{#1}%
  \def\temptrue{true}%
  \ifx\temparg\temptrue
    \bibbreakentrytrue
  \else
    \bibbreakentryfalse
  \fi
}

\AtBeginBibliography{%
  \ifbibbreakentry\else\interlinepenalty=10000\fi
}

% ------- 附录1 -------%
%--- 标题 ---%
\newcommand{\ftitle}[1]{%
  \begin{center}%
    \heiti\xiaoer #1
  \end{center}%
}

\newcommand{\fauthor}[1]{%
  \begin{center}%
    \parbox{\textwidth}{%
      \centering
      \songti\xiaosi
      \setlength{\baselineskip}{23.086043pt}%
      \selectfont
      #1
    }%
  \end{center}%
}

\newcommand{\thanksA}[1]{\textsuperscript{\hyperlink{affil:#1}{#1}}}
\newcommand{\thanksB}[1]{\noindent\hypertarget{affil:#1}{\textnormal{#1}}}

\newcommand{\corrauth}{\textsuperscript{\hyperlink{corrmark}{*}}}
\newcommand{\corrinfo}[1]{\noindent\hypertarget{corrmark}{\textsuperscript{*}通讯作者:#1}}

%------- 附录中文摘要 -------%
\newenvironment{cfabstract}{
    \chapter*{{
    \xiaoer
    \vspace*{-23.575pt}\\
    \hspace*{0.2em}摘\hspace{1em}要}
    \vspace*{-4.5pt}}
    \songti\xiaosi
    \setlength{\baselineskip}{23.086043pt}
    \thispagestyle{plain}
}

%--- 中文附录关键词 ---%
\def\cfkeywords#1{
    \vskip 1.95em
    \begin{spacing}{1} % 目测经验值,一般关键词不换行,影响不大
      \noindent\heiti\xiaosi\hspace*{-0.2em}\textbf{关键词:}\songti\xiaosi #1
    \end{spacing}
    \thispagestyle{plain}
}

%------- 附录英文摘要 -------%
\newenvironment{efabstract}{
    \chapter*{{
    \xiaoer
    \vspace*{-23.575pt}\\
    \textrm{\hspace{0.2em}\textbf{ABSTRACT}}}
    \vspace*{-4.5pt}}
    \xiaosi
    \setlength{\baselineskip}{23.086043pt}
    \thispagestyle{plain}
}

%--- 附录英文关键词 ---%
\def\efkeywords#1{
    \vskip 2em
    \begin{spacing}{1}
      \noindent\xiaosi\textbf{Key words:~}\xiaosi #1
    \end{spacing}
    \pagestyle{plain}
}

% ------- 计数器定义 -------%
\newcounter{fchapter}
\renewcommand{\thefchapter}{\arabic{fchapter}}
\newcounter{fsection}[fchapter]
\renewcommand{\thefsection}{\thefchapter.\arabic{fsection}}
\newcounter{fsubsection}[fsection]
\renewcommand{\thefsubsection}{\thefsection.\arabic{fsubsection}}
\newcounter{fsubsubsection}[fsubsection]
\renewcommand{\thefsubsubsection}{\thefsubsection.\arabic{fsubsubsection}}


% \fchapter定义
\newcommand{\fchapter}[1]{%
  \refstepcounter{fchapter}%
  \chapter*{\thefchapter\,\,\,#1}%
}

\makeatletter

% \fsection定义
\newcommand{\fsection}[1]{
  \refstepcounter{fsection}
  \section*{\thefsection\,\,\,#1}
}

% \fsubsection定义
\newcommand{\fsubsection}[1]{
  \refstepcounter{fsubsection}
  \ifdim\pagetotal>635pt % 跟着改下
    \ifdim\pagetotal<702pt
      \global\@nextpage@special@subsectiontrue
    \fi
  \fi
  \ifdim\pagetotal<0.1\topskip
    \thispagestyle{subsecpage}%
  \fi
  \oldsubsection*{\thefsubsection\,\,\,#1}
}

% \fsubsubsection定义
\newcommand{\fsubsubsection}[1]{
  \refstepcounter{fsubsubsection}
  \ifdim\pagetotal>620pt
    \ifdim\pagetotal<702pt
      \global\@nextpage@specialtrue
    \fi
  \fi
  \ifdim\pagetotal<0.1\topskip
    \thispagestyle{subsubsecpage}%
  \fi
  \oldsubsubsection*{\thefsubsubsection\,\,\,#1}
}

\makeatother

% ------- fequation编号定义 -------
\newcounter{fequation}[fchapter] % 每章重置
\renewcommand{\thefequation}{\thefchapter.\arabic{fequation}}

% ------- fequation环境 -------%
\newenvironment{fequation}{
  \refstepcounter{fequation}%
  \renewcommand{\theequation}{\thefequation}%
  \begin{equation}
}{
  \end{equation}
}

% ------- falign环境 -------%
\let\oldalign\align
\let\endoldalign\endalign
\newenvironment{falign}
  {\refstepcounter{fequation}%
   \renewcommand{\theequation}{\thefequation}%
   \oldalign}
  {\endoldalign}

% ------- ffigure计数器和编号格式 -------
\newcounter{ffigure}[fchapter]
\renewcommand{\theffigure}{\thefchapter.\arabic{ffigure}}

% ------- ffigure环境定义 -------
\newenvironment{ffigure}{
  \refstepcounter{ffigure}%
  \renewcommand{\thefigure}{\theffigure}%
  \begin{figure}
}{
  \end{figure}
}

% ------- fbiaoge计数器和编号格式 -------
\newcounter{fbiaoge}[fchapter]
\renewcommand{\thefbiaoge}{\thefchapter.\arabic{fbiaoge}}

% ------- fbiaoge命令定义 -------
\newcommand{\fbiaoge}[7][1.96]{%
  \refstepcounter{fbiaoge}%
  \renewcommand{\thetable}{\thefbiaoge}%
  \biaoge[#1]{#2}{#3}{#4}{#5}{#6}{#7}%
}

% ------- fsdbiaoge命令定义 -------
\newcommand{\fsdbiaoge}[6][1.96]{%
  \refstepcounter{fbiaoge}%
  \renewcommand{\thetable}{\thefbiaoge}%
  \sdbiaoge[#1]{#2}{#3}{#4}{#5}{#6}%
}

%------- 附录2 -------%
\RequirePackage{tocloft}
\makeatletter
\newcommand{\appendixtwo}{%
\chapter*{附录2:外文原文}\label{app2}%
\thispagestyle{app2}%

\ifx\@tocchaptertitletimes\@true
    \addtocontents{toc}{\vspace{1.755em}}%
    \addtocontents{toc}{%
    \protect\noindent
    \protect\hyperref[app2]{\sihao\heiti 附录\timesrm{2}:外文原文}%
    \protect\par
  }%
  \else
    \addtocontents{toc}{\vspace{1.755em}}%
    \addtocontents{toc}{%
    \protect\noindent
    \protect\hyperref[app2]{\sihao\heiti 附录{\heitiarabic{2}}:外文原文}%
    \protect\par
  }%
  \fi
}
\makeatother

%------- 致谢 -------%
\newcommand{\ack}{%
\chapter*{致\hspace{2em}谢}\label{ack}%
\thispagestyle{ack}
\addtocontents{toc}{\vspace*{1.755em}}%
\addtocontents{toc}{\hyperref[ack]{\noindent\heiti\sihao\hspace*{-1.05mm} 致谢}}%
}

%---------- 用户设置 ----------%
\InputIfFileExists{settings/cover}{}{} % 键入封面信息(必选)
\InputIfFileExists{settings/statement}{}{} % 键入“郑重声明”信息(必选)
\InputIfFileExists{settings/printmode}{}{} % 设置打印模式(必选)
\InputIfFileExists{settings/siunit}{}{} % 声明自定义单位(可选)
\InputIfFileExists{settings/typeset}{}{} % 设置全文排版细节(可选)
\InputIfFileExists{settings/bibliography}{}{} % 设置参考文献页的局部排版细节(可选)
\InputIfFileExists{settings/timesorheiti}{}{} % 设置黑体标题中的西文是否使用黑体(可选)
\InputIfFileExists{settings/footnotecounter}{}{} % 设置页末注计数行为(可选)
\InputIfFileExists{settings/hyperref}{}{} % 设置超链接格式(可选)
%    \end{macrocode}
%
% \iffalse
%</class>
% \fi
%
% \subsection{untilchapter 宏包}
%
% \textsf{untilchapter} 宏包用于页面样式管理,
% 特别是在章节前应用特定的页面样式。
%
% \iffalse
%<*package>
% \fi
%    \begin{macrocode}
\NeedsTeXFormat{LaTeX2e}
\ProvidesPackage{untilchapter}[2025/07/17 v1.0 wjh Apply pagestyle until...]
\RequirePackage{etoolbox}

\makeatletter
\newcommand{\usestyleuntil}[1]{%
  % 1. 切换到指定的页样式
  \pagestyle{#1}%
  % 2. 在出现目标命令时恢复 plain
  \pretocmd{\usestyleuntil}{\pagestyle{plain}}{}{%
    \PackageError{usestyleuntil}{Failed to patch \string\usestyleuntil}{}%
  }%
}
\makeatother
%    \end{macrocode}
%
% \iffalse
%</package>
% \fi
%
% \Finale
\endinput
